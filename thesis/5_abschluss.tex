\section{Ausblick}
\subsection{Mögliche Verbesserungen durch Einsatz von Angular statt AngularJS}
Angular bietet im Vergleich zum älteren AngularJS einige Verbesserungen im Hinblick auf automatisierte Testbarkeit, welche in diesem Kapitel oberflächlich betrachtet werden sollen.

Angular wird zusammen mit der Angular CLI ausgeliefert, einem Kommandozeilentool zur Vereinfachung häufig benötigter Abläufe. Der integrierte Assistent zur Generierung der Grundstruktur neuer Angular-Projekte bindet in diese auch eine Unterstützung für automatisierte Tests ein: ein konfiguriertes Karma sowie passende Ordnerstrukturen. Dieses kann umkonfiguriert werden, so dass auch die Nutzung von Mocha und Chai, statt dem standardmäßigen Jasmine, möglich ist. \cite{ng2-cli}

Angular bietet die \textit{Angular Testing Utilities}, wobei es sich um Klassen und Funktionen handelt, welche die Testdurchführung erleichtern. Sie ermöglichen es, ein \texttt{TestBed} zu erstellen, mit welchem ein Testmodul geschaffen wird, welches die zu testende Komponente enthält. Aus diesem Testmodul kann ein \texttt{TestFixture} kreiert werden, welches Zugriff auf die zu testende Komponente sowie auf das, der Komponente zugeordnete, DOM-Element bietet. Dies bietet im Vergleich zu Tests mit ngMock in AngularJS den Vorteil einer besseren Lesbarkeit und dem einfacheren Zugriff auf interne Funktionen und Eigenschaften. Ein beispielhafter Test unter Nutzung der Angular Testing Utilities findet sich in Listing \ref{lst:ng2-test}.  \cite{ng2-test}

Zusammenfassend lässt sich sagen, dass die Realisierung von automatisierten Tests in neueren Versionen von Angular mit weniger Problemen und deutlich einfacher möglich ist, so dass sich mögliche Anfangshindernisse und -schwellen verringern und so den Einstieg eines Entwicklers in automatisierte Tests erleichtern. 

\begin{figure}[H]
	\lstinputlisting[caption={Beispielhafter Komponententest einer \textit{Component} in Angular; geschrieben in TypeScript (aus  \cite{ng2-test})}, label=lst:ng2-test]{lst/ng2-test.js}
\end{figure}

\subsection{Fazit}
Es konnte eine Softwareauswahl gefunden werden, mit welcher sich automatisierte Komponententests für AngularJS-Anwendungen im Commerzbank-Konzern realisieren lassen. Auch konnte durch die prototypische Testentwicklung für das Projekt Gefährdungsbeurteilungen gezeigt werden, dass diese Auswahl problemlos funktioniert, einfach anzuwenden ist und sich auch in bestehende Projekte integrieren lässt. Dadurch sollte die Eintrittsschwelle und der initiale Aufwand einer Einbindung in existierende und neue Projekte gesunken sein. Es ist nun zu hoffen, dass die ausgewählte Software und überhaupt die Idee automatisierter Tests Anklang findet und so sukzessive die Testabdeckung gesteigert und eine höhere Softwarequalität erreicht werden kann. Als fördernde Maßnahme ist hierzu eine Vorstellung der Ergebnisse dieser \titleDocument{} und eine Einführung in die grundlegenden Vorgehensweisen zur Entwicklung automatisierter Tests in AngularJS-Anwendungen für die Kolleginnen und Kollegen in der \domain{} durch den Autor angedacht.

Als letztlich kleiner Misserfolg kann die Untersuchung der Möglichkeiten der Durchführung automatisierter End-To-End-Tests angesehen werden. Sämtliche gefundenen und untersuchten Tools scheitern an den Anforderungen, welche sich durch regulatorisch und historisch bedingte Regelwerke, Prozesse und technische Systeme im Commerzbank-Konzern ergeben. Zukünftig soll versucht werden, diese Hindernisse zu beseitigen, zu ändern oder durch geeignete Maßnahmen zu überwinden, um die Durchführung automatisierter End-To-End-Tests zu ermöglichen und eine weitere Verbesserung der Effektivität und Qualität herbeizuführen.

Alles in Allem wurden die Ziele dieser Arbeit erfüllt und gute Grundlagen für die zukünftige Softwareentwicklung in der \domain{} und Commerzbank AG geschaffen.
