Der Test von Software dient dazu, mögliche Fehler aufzudecken und dadurch die Qualität zu erhöhen. Der Nachweis von Fehlerfreiheit ist unmöglich, daher muss der Testaufwand verhältnismäßig zum Ergebnis sein\cite[][14\psqq]{spillner}. 

Beim Testen werden üblicherweise vier Teststufen unterschieden\cite[][42\psq]{spillner}:
\begin{itemize}
	\item Komponententest
	\item Integrationstest
	\item Systemtest
	\item Abnahmetest
\end{itemize}
Bei Komponenten-, Integrations- sowie bedingt bei Systemstests handelt es sich um Entwicklertests, weshalb diese im Rahmen dieser \titleDocument{} relevant sind. Der Abnahmetest wird daher nicht betrachtet.

\subsubsection{Komponententest}
Ein Komponententest überprüft die einzelnen Bausteine der entwickelten Software erstmalig und unabhängig von anderen Bausteinen. Es wird überprüft, ob die Komponente den Anforderungen sowie dem definierten Softwaredesign entspricht. Außerdem kann auch der Quellcode analysiert und zur Erstellung der Testfälle herangezogen werden; dann handelt es sich um einen \textit{Whitebox-Test}.\cite[][44]{spillner} In JavaScript ist die kleinste testbare Komponente üblicherweise eine Funktion\cite{smashing-unit}. Die zu testende Komponente muss nicht zwingend atomar sein, d.\,h. die Funktion kann aus weiteren Funktionen zusammengesetzt sein, jedoch sollte nur die komponenteninterne Funktionsweise getestet werden\cite[][45]{spillner}. Beim Test von AngularJS-Anwendungen sind die Komponenten beispielsweise Controller, Services, Direktiven und Filter.

Ein Komponententest hat spezifische Testziele. Das Wichtigste ist die Sicherstellung, dass die Funktion der Anforderung in der Spezifikation entspricht. Hierdurch wird die Komponente wie gefordert mit anderen Komponenten zusammenarbeiten und somit in die Gesamtsoftware integriert werden können. Wichtig ist auch der Test auf Robustheit: Bei falschem Aufruf, also einem Verstoß gegen die Vorbedingungen, sollte die Komponente sinnvoll reagieren und den Fehler abfangen. Testfälle lassen sich in Positiv- und Negativtests unterteilen: Positivtests sind die Überprüfung von vorgesehenem Verhalten der Komponente, Negativtests der Test von nicht vorgesehenen, unzulässigen oder explizit ausgeschlossenen Sonderfällen.\cite[][48]{spillner}.

Im Komponententest können auch nicht funktionale Qualitätseigenschaften getestet werden. Zu nennen sind hier beispielsweise Speicherverbrauch oder Antwortzeit, sowie statische Tests auf Wartbarkeit, wie beispielsweise vorhandene Quelltextkommentare oder die Einhaltung von Programmierrichtlinien.\cite[][49\psq]{spillner}.

\subsubsection{Integrationstest}
Der Integrationstest folgt nach den Komponententests und basiert auf getesteten Komponenten. Diese Komponenten werden zu größeren Komponenten oder Teilsystemen zusammengesetzt. Der Integrationstest dient dann dazu zu überprüfen ob alle Einzelteile korrekt zusammenarbeiten und soll Fehler in Schnittstellen und im Zusammenspiel aufdecken. \cite[][52\psq]{spillner} Beispielsweise kann in einem Integrationstest auch die Anbindung an externe Komponenten, wie Datenbanken oder REST-APIs überprüft werden. Diese werden im Komponententest durch \textit{Mocks} ersetzt und emuliert.

Somit ist das aufdecken von Schnittstellenfehlern ein Testziel des Integrationstest, zum Beispiel wegen von der Spezifikation abweichender Schnittstellen. Außerdem ist ein Testziel unerwünschte Wechselwirkungen zwischen den Einzelkomponenten aufzudecken, welche das Zusammenspiel unmöglich machen. \cite[][56]{spillner}

Der Integrationstest ist jedoch kein Ersatz für den Komponententest, da er mit Nachteilen verbunden ist. Es ist schwer bis unmöglich, die tatsächliche Fehlerursache herauszufinden, da oft nicht klar ist in welcher Teilkomponente der Fehler aufgetreten ist, sondern dieser sich nur in einem abweichenden Gesamtverhalten äußert. Manche Fehler werden möglicherweise gar nicht gefunden, da regelmäßig kein vollumfänglicher Zugriff auf Einzelkomponenten besteht.\cite[][57]{spillner}

Es existieren verschiedene Integrationsstrategien, die Auswirkungen auf die Integrationstests haben:\cite[][59\psq]{spillner}
\begin{itemize}
	\item Bei der Top-Down-Integration beginnt der Test mit der obersten Systemkomponente, von der alle Anderen aufgerufen werden. Sukzessive werden die weiteren Komponenten von oben nach unten integriert und getestet, wobei die untergeordneten Komponenten zunächst durch Platzhalter ersetzt werden. Vorteilhaft ist, dass keine aufwändigen Testtreiber zum Aufruf benötigt werden. Jedoch müssen Platzhalterkomponenten implementiert werden, was einen zusätzlichen Overhead beim Test bedeuten kann.
	\item Bei der Bottom-up-Integration werden zunächst die unteren, atomaren Komponenten integriert und getestet. Erst nach und nach werden größere Teilsysteme aus getesteten Komponenten integriert. Der Vorteil hierbei ist, dass keine Platzhalter implementiert werden müssen. Jedoch müssen hier aufwändige Testtreiber erstellt werden, welche die obergeordneten, aufrufenden Komponenten emulieren.
	\item Bei der Ad-hoc-Integration werden Komponenten integriert, sobald sie fertiggestellt sind. Nachteilig hierbei ist, dass sowohl Platzhalter als auch Testtreiber implementiert werden müssen. Allerdings bietet sich ein Zeitgewinn, da jede Komponente so früh wie möglich integriert wird.
	\item Bei der wenig empfehlenswerten Big-Bang-Integration werden alle Komponenten auf einmal integriert. Sie bietet ausschließlich Nachteile: Es wird Zeit verschwendet, da bis zur Fertigstellung der letzten Komponente gewartet wird. Auch treten alle Fehlerwirkungen gesammelt auf, so dass es schwierig ist, die Fehler zu finden.
\end{itemize}

\subsubsection{Systemtest}
Der Systemtest ist der finale Entwicklertest nach den abgeschlossenen Integrationstests. Getestet wird das gesamte System, möglichst in einer produktionsnahen Umgebung. Es sind keine Testtreiber oder Platzhalter mehr vorhanden, stattdessen wird überall die finale Hard- und Software genutzt. Das Testziel ist die Validierung, ob alle funktionalen und nicht-funktionalen Anforderungen erfüllt werden.\cite[][60\psqq]{spillner}

Aufgrund der erforderlichen Produktionsnähe sowie der erforderlichen Datenbasis kann der Systemtest nur bedingt als Entwicklertest gesehen werden\cites[236]{roitzsch}[]{oose}. Jedoch kann und sollte ein einfacher Systemtest auch von Entwicklern durchgeführt werden. Es bietet sich hier die Durchführung von End-To-End-Tests an, mittels derer das System von der Benutzeroberfläche bis zur untersten Komponentenschicht getestet werden kann\cite{softwaresanierung}.