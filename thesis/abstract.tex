\section*{Abstract}
In dieser \titleDocument{} werden die Möglichkeiten zur Realisierung automatisierter Tests von AngularJS-Anwendung untersucht, hierbei wird speziell auf die Erfordernisse im Commerzbank-Konzern geachtet.

Zunächst wird eine Einführung in die allgemeinen Themen Testen, Angular und Node.js gegeben und Anforderungen an Testframework und -tools definiert. Anschließend werden die verschiedene Softwares untersucht und mit Blick auf die Erfüllung der gestellten Anforderungen bewertet. Eine Kombination von Karma, Mocha, Chai und Protractor, sowie Sinon, ngMock und Istanbul, wird für den Einsatz im Commerzbank-Konzern als am geeignetsten betrachtet und ausgewählt.

Eine prototypische Realisierung von Tests im Commerzbank-Projekt \textit{GFB} mit der Auswahl zeigt, dass die Realisierung von Komponententests mit Karma, Mocha und Chai problemlos funktioniert und einfach anzuwenden ist. Es zeigt sich jedoch auch, dass eine Durchführung von automatisierten End-To-End-Tests an Commerzbank-Arbeitsplätzen aufgrund verschiedener regulatorischer Vorgaben und Einschränkungen unmöglich ist und auch die ausgewählte Software Protractor für diesen Zweck nicht funktioniert.

Abschließend wird ein Ausblick auf mögliche Verbesserungen bei automatisierten Tests durch den Einsatz neuerer Versionen von Angular vorgenommen und ein positives Fazit gezogen.