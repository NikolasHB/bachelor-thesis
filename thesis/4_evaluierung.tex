\section{Evaluierung}
Im folgenden Kapitel wird die getroffene Softwareauswahl evaluiert, indem für ein gegebenes, bereits existierendes AngularJS-Projekt im Commerzbank-Konzern prototypisch Tests implementiert werden. Das Projekt konnte nicht beliebig ausgesucht werden, sondern wurde von der \domain zur Verfügung gestellt. Hierbei werden nicht Tests für alle Komponenten und Funktionalitäten der Anwendung entwickelt, sondern exemplarisch nur das Minimum, welches zur Feststellung einer problemlosen Einsatzmöglichkeit ausreicht.

\subsection{Projektbeschreibung}
Das Projekt GFB stellt ein digitales Formular bereit, mittels welchem Gefährdungsbeurteilungen erfasst werden können. Dies sind Dokumente, welche beispielsweise von Fachkräften für Arbeitssicherheit nach Betriebsbegehungen erstellt werden und in denen festgestellte Mängel aufgeführt werden. \cite{gfb}

Die Anwendung ist eine Single-Page-Application und wurde unter Nutzung von AngularJS entwickelt. Bereitgestellt wird sie in einer Microsoft SharePoint-Umgebung, wodurch sich einige Besonderheiten ergeben: z.\,B. die Nutzung von SOAP zur Abfrage von WebServices. Durch die SharePoint-Umgebung funktioniert die Anwendung ausschließlich im Internet Explorer und nicht im zweiten Commerzbank-Standardbrowser Firefox. Für die Anwendung existieren keine automatisierten Tests; jegliches Testen wird bisher auf der Oberfläche im Testsystem durchgeführt. \cite{gfb}

\subsection{Implementierung}
Die Implementierung der Tests erfolgt in drei Iterationen: Zunächst wird eine Testumgebung, bestehend aus Karma, Mocha, Chai, Sinon und Istanbul, aufgesetzt und gezeigt, dass diese funktioniert. Anschließend werden prototypische Komponententests für alle verschiedenen Komponententypen (Controller, Factory, Filter, Direktive) in AngularJS implementiert. Abschließend werden Systemtests als End-To-End-Tests umgesetzt; auch dies geschieht prototypisch und hat keine hundertprozentige Abdeckung zum Ziel.

\newpage
\subsubsection{Testumgebung}
Das bestehende Projekt konnte problemlos in ein npm-Projekt überführt werden. Alle existierenden Quelldateien wurden in einen Unterordner \texttt{src} verschoben und folgende Ordnerstruktur geschaffen:
\begin{figure}[H]
\dirtree{%
	.1	/.
	.2	coverage\DTcomment{Speicherort für Code-Coverage-Reports}.
	.3	html.
	.3	lcov.
	.2	node\_modules\DTcomment{Speicherort für npm-Module}.
	.2	src\DTcomment{Originaldateien der GFB-Anwendung}.
	.2	test.
	.3	spec.
	}
\end{figure}

Über npm konnten alle für die Ausführung von Tests benötigten Komponenten installiert werden: Karma mitsamt diverser Plugins, Mocha, Chai, Istanbul, angular-mocks, Sinon und Protractor. Auch hierbei traten keinerlei Probleme auf. Die fertige \texttt{package.json}-Datei ist in Listing \ref{lst:gfb-package} im Anhang zu finden.

Zur Verifizierung der Einsatzbereitschaft der installierten Software wurde eine einfache Testsuite geschrieben, welche eine Assertion durchführt und ein Spy mit Sinon erstellt. Für die Verifizierung von ngMock sowie die korrekte Einbindung aller originalen Projektdateien wurde eine weitere Testsuite erstellt, in welcher in zwei Testfällen ein Service und ein Filter, welche im Originalprojekt definiert sind, von ngMock injiziert werden. Die Testsuites finden sich in Listing \ref{lst:gfb-valid1} und \ref{lst:gfb-valid2} im Anhang. 

Karma wurde so konfiguriert, dass es automatisch den Internet Explorer startet und diesen zur Ausführung der Tests verwendet. Hierbei trat das Problem auf, dass der Browser zwar gestartet wird, jedoch von Karma der erfolgreiche Start nicht korrekt erkannt wird. Laut Dokumentation des \texttt{karma-ie-launcher} führt Starten des Internet Explorers im \textit{no add-ons mode} zur Lösung dieses Problems \cite{karma-ie-launcher}. Nach Vornehmen dieser Änderung wird der Internet Explorer korrekt durch Karma gestartet und die Tests in ihm ausgeführt.

Zur Ermittlung der Code-Coverage wurde das Plugin \texttt{karma-coverage} hinzugefügt, durch welches Karma das Code-Coverage-Tool Istanbul startet. Dieses wurde so konfiguriert, dass die erstellen Berichte über die Codeabdeckung im Unterordner \texttt{coverage} in zwei Formaten abgelegt werden. Zur manuellen Ansicht im Browser dient das HTML-Format, welches eine durchklickbare Seite bereitstellt, in welcher alle erforderlichen Statistiken angezeigt werden; beispielhafte Screenshots dieser Anzeige finden sich in Abbildung \ref{abb:code-cov-1} und \ref{abb:code-cov-2} im Anhang. 

Die fertiggestellte Konfigurationsdatei für Karma ist in Listing \ref{lst:gfb-karma} im Anhang zu finden. Bei Ausführung des Befehls \texttt{karma start} treten keine Fehler auf; alle Testfälle werden als bestanden markiert. Es ist daher davon auszugehen, dass die eingerichtete Testumgebung für Komponententests funktioniert. 	

\subsubsection{Komponententests}
Nach der Feststellung, dass das Testsystem korrekt eingerichtet wurde, werden prototypisch Tests für alle in der Anwendung vorhandenen AngularJS-Komponenententypen entwickelt.

\paragraph{Controller}
In der Anwendung existiert ein Controller, der \texttt{OverviewCtrl}, welcher somit für Tests ausgewählt wird. Aufgrund der hohen Komplexität der initialisierenden Controller-Funktion eignet diese sich nicht für die Erstellung eines Komponententest. Daher wird eine zum Controller gehörende Scope-Funktion getestet: \texttt{\$scope.validatebegehungsdatum}. Diese validiert, ob zu einer Begehung ein Datum eingegeben wurde und gibt dann true zurück; ansonsten false. Einer der entwickelten Testfälle findet sich in Listing \ref{lst:gfb-overviewCtrlExtract}; es handelt sich hierbei um einen Auszug aus der im Anahng in Listing \ref{lst:gfb-overviewCtrl} zu findenden gesamten Testsuite.

\begin{figure}[H]
	\lstinputlisting[caption={Testfall aus der zum OverviewCtrl gehörenden Testsuite (s. Listing \ref{lst:gfb-overviewCtrl})}, label=lst:gfb-overviewCtrlExtract]{lst/gfb/test/spec/auszugOverviewCtrl_spec.js}
\end{figure}

Zugriff auf den Controller wird hierbei durch Nutzung des standardmäßigen AngularJS-Services \texttt{\$controller} erlangt \cite{angular-controllerAPI}, welcher in der \texttt{beforeEach}-Funktion durch ngMock in den Test injiziert wird. Ein neuer Scope wird mit der Methode \texttt{\$new} des Root-Scopes erzeugt und dem Controller zur Nutzung übergeben. Die Variable \texttt{initialData} ist ein in der \texttt{beforeEach}-Funktion erzeugtes Dummyobjekt und wird vom Controller zwingend benötigt. Nach den Vorbereitungen kann die zu testende Scope-Funktion aufgerufen und der Rückgabewert mit der Erwartung verglichen werden. Bei Ausführung zeigt sich, dass der Test korrekt ausgeführt wird, keine Fehler ausgegeben und alle definierten Testfälle als bestanden markiert werden.

\paragraph{Factory}
In der Anwendung existieren viele Factories, von denen jedoch die meisten als externe Software zugeliefert wurden. Für den Test wird daher eine Factory gewählt, welche projektintern entwickelt wurde. Diese sind in der \texttt{svc\_obs.js} zu finden und beinhalten die hauptsächliche Geschäftslogik der Anwendung. Die Factory \texttt{cls\_betriebsratsbereich} bildet die Datenstruktur eines Betriebsratsbereiches ab und bietet eine Funktion, diese Datenstruktur aus einem XML-Element zu befüllen. Ein solches Element ist nach folgendem Muster aufgebaut: \texttt{<item ows\_ID=1 ows\_Title='Betriebsrat Bremen' ows\_Sortierung=1 />}. Ein Auszug aus der kompletten Testsuite (s. Listing \ref{lst:gfb-svc_obs} im Anhang) findet sich in Listing \ref{lst:gfb-svc_obsExtract}.


\begin{figure}[H]
	\lstinputlisting[caption={Testfall aus der zu svc\_obs gehörenden Testsuite (s. Listing \ref{lst:gfb-svc_obs})}, label=lst:gfb-svc_obsExtract]{lst/gfb/test/spec/auszugsvc_obs_spec.js}
\end{figure}

Vor Ausführung jedes Testfalls wird in der \texttt{beforeEach}-Funktion die zu testende Factory injiziert und unter Nutzung des Operators \texttt{new} instanziiert. Beim hierbei übergebenen \texttt{svc\_sp} handelt es sich um eine weitere Factory, welche Methoden zur SOAP-Kommunikation mit dem unterliegenden SharePoint bietet. Der oben beschriebene Aufbau des XML-Elementes wird hier als Dummyobjekt umgesetzt, welches die intern zum Abruf von Attributen genutzte Funktion \texttt{attr} als Sinon-Stub enthält. Dieses Stub ist so konfiguriert, dass es die jeweiligen Testwerte zurückgibt. Nach Aufruf der zu testenden Funktion \texttt{loadfromxmlitem} können die im Objekt enthaltenen Werte mit den Erwarteten verglichen werden. Die Ausführung der implementierten Testsuite ergibt keine Fehler; alle Tests werden korrekt ausgeführt und als bestanden markiert.

\paragraph{Direktive}
Für den Test einer Direktive wird \texttt{dir\_filHb} ausgewählt, welche eine mit einem Button kombinierte Textbox zur Eingabe einer Fillialkennnummer realisiert. Für die Textbox ist in der Direktive eine Funktion definiert, welche aufgerufen wird, wenn der Fokus die Textbox verlässt. Die Funktion validiert die Eingabe auf das erlaubte Format von fünf Ziffern; bei fehlerhafter Eingabe wird ein Alert ausgegeben und die CSS-Klasse \texttt{ng-invalid-input} gesetzt. Die vollständige Testsuite findet sich in Listing \ref{lst:gfb-dir_filHb} im Anhang, ein Auszug in Form eines einzelnen Testfalls in Listing \ref{lst:gfb-dir_filHbExtract}.

\begin{figure}[H]
	\lstinputlisting[caption={Testfall aus der zu dir\_filHb gehörenden Testsuite (s. Listing \ref{lst:gfb-dir_filHb})}, label=lst:gfb-dir_filHbExtract]{lst/gfb/test/spec/auszugdir_filHb_spec.js}
\end{figure}

Zugriff auf die zu testende Direktive wird hierbei über den \texttt{\$compile}-Service von AngularJS erlangt, welcher einen übergebenen HTML-String in ein Template kompiliert \cite{angular-compile}. Dieses Template wird dann zusammen mit einem übergebenen Scope zu einem Element verbunden. Zugriff auf die im Scope der Direktive vorhandenen Funktionen ist über die Funktion \texttt{isolateScope} möglich, was die Ausführung der zu testenden Funktion \texttt{myBlur} (Fokus verlässt die Textbox) ermöglicht. Die Ausführung der Testsuite zeigt keine Fehler; alle Tests werden als bestanden markiert und korrekt ausgeführt.

\paragraph{Filter}
In der Anwendung existiert der Filter \texttt{unique}, welcher aus einer übergebenen Liste alle gleichen Einträge entfernt und somit eine eindeutige Liste ohne Dopplungen erzeugt. Für die Überprüfung der Gleichheit zweier Eintrage wird jedoch nicht das gesamte Element verglichen, sondern nur ein einzelner Schlüsselindex, so dass beispielsweise die Elemente \texttt{[1, 'Bremen']} und \texttt{[1, 'Hamburg']} bei Betrachtung des nullten Schlüsselindex als gleich betrachtet werden. Ein Testfall für den \texttt{unique}-Filter findet sich in Listing \ref{lst:gfb-uniqueExtract}; es handelt sich hierbei um einen Auszug aus der vollständigen Testsuite aus Listing \ref{lst:gfb-unique} im Anhang.

\begin{figure}[H]
	\lstinputlisting[caption={Testfall aus der zum unique-Filter gehörenden Testsuite (s. Listing \ref{lst:gfb-unique})}, label=lst:gfb-uniqueExtract]{lst/gfb/test/spec/auszugunique_spec.js}
\end{figure}

Die zu testende Filter-Funktion kann mithilfe des injizierten AngularJS-Services \texttt{\$filter} in einer Variable gespeichert werden  \cite{angular-filter}. Die so erlangte Filterfunktion kann dann in den einzelnen Testfällen aufgerufen und das Ergebnis mit der Erwartung verglichen werden. Bei Ausführung zeigt sich, dass der Test korrekt ausgeführt wird, keine Fehler ausgegeben und alle definierten Testfälle als bestanden markiert werden.

\subsubsection{End-To-End-Tests}
Für die Durchführung von End-To-End-Tests musste das installierte Protractor eingerichtet werden. Hierzu wurde gemäß der Dokumentation  \cite{protractor-index} der Befehl \texttt{webdriver-manager update} aufgerufen. Auch nach Einstellung des zu verwendenden Proxyservers konnten die zu installierenden Abhängigkeiten nicht aufgelöst werden, da die Verbindung vom Proxyserver zurückgewiesen wird. Ohne die vorhandenen ausführbaren Dateien des Selenium-Servers (\texttt{selenium-server-standalone.jar}) und des Internet-Explorer-Treibers (\texttt{IEDriverServer.exe}) ist Protractor nicht funktionsfähig. Zur Feststellung der grundsätzlichen Funktionsfähigkeit wurden die benötigten Dateien über verschiedene, vom Commerzbank-Standard abweichende, Wege bezogen und einzeln im Ordner abgelegt. Nach dieser manuellen Installation konnte der Selenium-Server erfolgreich gestartet werden. 

Zum Test der Einsatzbereitschaft wurde Protractor konfiguriert: Die Tests sollen im Internet Explorer ausgeführt werden; als Testframework soll Mocha genutzt werden. Die fertiggestellte Konfigurationsdatei ist in Listing \ref{lst:gfb-protractorConf} im Anhang zu finden. Als auszuführender Test wurde der Beispieltest aus  \cite{protractor-index} verwendet, welcher für die Nutzung mit Mocha abgewandelt wurde (s. Listing \ref{lst:gfb-protractorTest} im Anhang).

Bei Ausführung von Protractor zeigte sich, dass der Start des Internet Explorers nicht möglich ist und einen Fehler erzeugt. Zur Problemlösung wird unter Anderem die Deaktivierung des \textit{Geschützten Modus} in allen Zonen im Internet Explorer empfohlen (vgl.  \cite{bug-ie-1, bug-ie-2}). Das Zutreffen dieser Aussage konnte in einem lokalen System außerhalb des Commerzbank-Netzwerks validiert werden. Jedoch ist die Deaktivierung des Geschützten Modus' im Internet Explorer eines Commerzbank-Arbeitsplatzes nicht möglich (s. Abbildung \ref{abb:coba-ie} im Anhang).

Aufgrund der aufgeführten Problematiken ist ein Einsatz von Protractor im Commerzbank-Netz als unmöglich zu bewerten.

\subsection{Auswertung}
Zusammenfassend zeigt sich bei der Auswertung der Evaluierung der Softwareauswahl ein gemischtes Bild. Die Auswahl von Software für die Durchführung von Komponententests kann als erfolgreich angesehen werden. Sie funktioniert im Commerzbank-Umfeld problemlos.

Anders stellt sich dies bei der Auswahl für End-To-End-Tests dar: Protractor ist für den Einsatz im Commerzbank-Umfeld nicht geeignet, auch wenn in Abschnitt \ref{sec:auswahlentscheidung} alle Anforderungen als erfüllt gewertet sind. Die zur Verfügung stehenden Alternativen Intern und PhantomJS/CasperJS eignen sich auch nicht für einen Einsatz. Intern basiert genau wie Protractor auf Selenium und würde somit die gleichen Probleme mit der Konfiguration des Internet Explorers hervorrufen. Die Kombination aus PhantomJS und CasperJS hingegen benötigt für die Ausführung zwingend Python, was in der Commerzbank nicht zur Verfügung steht.