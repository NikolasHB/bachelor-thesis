\section{Motivation}\label{einleitung}
Die Commerzbank AG ist laut \textcite[][2\psqq]{handelsblatt:commerzbank} Deutschlands zweitgrößtes Finanzinstitut. Als solches stellen sich für ihre IT besondere Herausforderungen an Daten-, Ausfallsicherheit und Stabilität. Regulatorische Vorgaben durch die Bundesanstalt für Finanzdienstleistungsaufsicht\cite{bafin-banken}, interne Programmierrichtlinien\cite{coba-programmierrichtlinienAllgemein, coba-programmierrichtlinienJavaScript}, das Book of Standards \cite{coba-bookOfStandards} und Weitere führen zu einem trägen und wenig innovativem IT-Umfeld. Somit ist es nicht überraschend, dass zur Entwicklung von Webanwendungen noch immer auf alte und etablierte Technologien wie JavaServer Pages und jQuery zurückgegriffen wird\todo{Quelle!}.

Viele Unternehmen nutzen bereits seit einigen Jahren Angular als Technologiebasis für clientseitige Webanwendungen, beispielsweise \textit{Gmail}, \textit{PayPal} oder \textit{Youtube}\todo{quellen}. In der Commerzbank wurde Angular bisher nicht berücksichtigt; erst in jüngerer Vergangenheit wird es in vereinzelten Projekten eingesetzt. Hierbei werden automatisierte Entwicklertests in JavaScript aufgrund von Unwissenheit über die Möglichkeiten meist stiefmütterlich behandelt. Die Anwendungen werden stattdessen per Hand im Webbrowser getestet. Die sich hieraus ergebende Testabdeckung steht im Gegensatz zu den Anforderungen an Sicherheit und Stabilität im Bankenumfeld.

In dieser \titleDocument~ sollen daher die Möglichkeiten zur Realisierung von automatisierten Tests in AngularJS-Webanwendungen untersucht werden. \todo{...?}