% allgem. Dokumentenformat
\documentclass[a4paper,12pt,headsepline]{scrartcl}
%Variablen welche innerhalb der gesamten Arbeit zur Verfügung stehen sollen
\newcommand{\titleDocument}{Bachelorthesis}
\newcommand{\subjectDocument}{im Studiengang Duales Studium Informatik}
\newcommand{\domain}{Commerz Systems GmbH}

% weitere Pakete
% Grafiken aus PNG Dateien einbinden
\usepackage{graphicx}

% Text neben Grafik
\usepackage{wrapfig}

% Text neben Grafik
\usepackage{censor}

% Deutsche Sonderzeichen benutzen 
%\usepackage{ngerman}

% deutsche Silbentrennung
\usepackage[ngerman]{babel}

% Eurozeichen einbinden
%\usepackage[right]{eurosym}

% Umlaute unter UTF8 nutzen
\usepackage[utf8]{inputenc}

% Zeichenencoding
\usepackage[T1]{fontenc}

\usepackage{lmodern}
\usepackage{fix-cm}

% floatende Bilder ermöglichen
%\usepackage{floatflt}

% mehrseitige Tabellen ermöglichen
\usepackage{longtable}

\usepackage{adjustbox}
\usepackage{array}
\usepackage{booktabs}
\usepackage{multirow}
% Unterstützung für Schriftarten
%\newcommand{\changefont}[3]{ 
%\fontfamily{#1} \fontseries{#2} \fontshape{#3} \selectfont}

% Packet für Seitenrandabständex und Einstellung für Seitenränder
\usepackage{geometry}
\geometry{left=3.5cm, right=2cm, top=2.5cm, bottom=2cm}

% Paket für Boxen im Text
\usepackage{fancybox}

% bricht lange URLs "schoen" um
\usepackage[hyphens,obeyspaces,spaces]{url}

% Paket für Textfarben
\usepackage{color}

% Mathematische Symbole importieren
\usepackage{amssymb}

% auf jeder Seite eine Überschrift (alt, zentriert)
%\pagestyle{headings}

% erzeugt Inhaltsverzeichnis mit Querverweisen zu den Kapiteln (PDF Version)
\usepackage[bookmarksnumbered,pdftitle={\titleDocument},hyperfootnotes=false]{hyperref} 
%\hypersetup{colorlinks, citecolor=red, linkcolor=blue, urlcolor=black}
%\hypersetup{colorlinks, citecolor=black, linkcolor= black, urlcolor=black}

% neue Kopfzeilen mit fancypaket
\usepackage{fancyhdr} %Paket laden
\pagestyle{fancy} %eigener Seitenstil
\fancyhf{} %alle Kopf- und Fußzeilenfelder bereinigen
\fancyhead[L]{\nouppercase{\leftmark}} %Kopfzeile links
\fancyhead[C]{} %zentrierte Kopfzeile
\fancyhead[R]{\thepage} %Kopfzeile rechts
\renewcommand{\headrulewidth}{0.4pt} %obere Trennlinie
%\fancyfoot[C]{\thepage} %Seitennummer
%\renewcommand{\footrulewidth}{0.4pt} %untere Trennlinie

% für Tabellen
\usepackage{array}

% Runde Klammern für Zitate
%\usepackage[numbers,round]{natbib}

% Festlegung Art der Zitierung - Havardmethode: Abkuerzung Autor + Jahr
\usepackage[backend=biber,style=alphabetic]{biblatex}
\usepackage[babel,german=guillemets]{csquotes}
%\bibliographystyle{alphadin}
%\bibliography{BachelorArbeit}
\addbibresource{BachelorArbeit.bib}

% Schaltet den zusätzlichen Zwischenraum ab, den LaTeX normalerweise nach einem Satzzeichen einfügt.
\frenchspacing

% Paket für Zeilenabstand
\usepackage{setspace}
\usepackage{todonotes}
\usepackage{float}

% für Bildbezeichner
\usepackage{capt-of}

% für Stichwortverzeichnis
\usepackage{makeidx}

% für Listings
% \usepackage{listings}
% \lstset{numbers=left, numberstyle=\tiny, numbersep=5pt, keywordstyle=\color{black}\bfseries, stringstyle=\ttfamily,showstringspaces=false,basicstyle=\footnotesize,captionpos=b}
% \lstset{language=java}

\usepackage{listings}
%\lstset{numbers=left, numberstyle=\tiny, numbersep=5pt, keywordstyle=\color{black}\bfseries, stringstyle=\ttfamily,showstringspaces=false,basicstyle=\footnotesize,captionpos=b}
\usepackage{color}

% Blocksatz?
\usepackage[none]{hyphenat}
\usepackage{dirtree}

\definecolor{mygreen}{rgb}{0,0.6,0}
\definecolor{mygray}{rgb}{0.5,0.5,0.5}
\definecolor{mymauve}{rgb}{0.58,0,0.82}

\lstset{ %
	backgroundcolor=\color{white},   % choose the background color; you must add \usepackage{color} or \usepackage{xcolor}
	basicstyle=\footnotesize,        % the size of the fonts that are used for the code
	breakatwhitespace=false,         % sets if automatic breaks should only happen at whitespace
	breaklines=true,                 % sets automatic line breaking
	captionpos=b,                    % sets the caption-position to bottom
	commentstyle=\color{mygreen},    % comment style
	deletekeywords={...},            % if you want to delete keywords from the given language
	escapeinside={\%*}{*)},          % if you want to add LaTeX within your code
	extendedchars=true,              % lets you use non-ASCII characters; for 8-bits encodings only, does not work with UTF-8
	frame=single,	                   % adds a frame around the code
	keepspaces=true,                 % keeps spaces in text, useful for keeping indentation of code (possibly needs columns=flexible)
	keywordstyle=\color{blue},       % keyword style
	language=Octave,                 % the language of the code
	otherkeywords={*,...},           % if you want to add more keywords to the set
	numbers=left,                    % where to put the line-numbers; possible values are (none, left, right)
	numbersep=5pt,                   % how far the line-numbers are from the code
	numberstyle=\tiny\color{mygray}, % the style that is used for the line-numbers
	rulecolor=\color{black},         % if not set, the frame-color may be changed on line-breaks within not-black text (e.g. comments (green here))
	showspaces=false,                % show spaces everywhere adding particular underscores; it overrides 'showstringspaces'
	showstringspaces=false,          % underline spaces within strings only
	showtabs=false,                  % show tabs within strings adding particular underscores
	stepnumber=2,                    % the step between two line-numbers. If it's 1, each line will be numbered
	stringstyle=\color{mymauve},     % string literal style
	tabsize=2,	                   % sets default tabsize to 2 spaces
	title=\lstname                   % show the filename of files included with \lstinputlisting; also try caption instead of title
}

% Indexerstellung
\makeindex

% Abkürzungsverzeichnis
\usepackage[german]{nomencl}
\let\abbrev\nomenclature

% Abkürzungsverzeichnis LiveTex Version
\renewcommand{\nomname}{Abkürzungsverzeichnis}
\setlength{\nomlabelwidth}{0.01\hsize}
\renewcommand{\nomlabel}[1]{#1 \dotfill}
\setlength{\nomitemsep}{-\parsep}
\makenomenclature

%\makeglossary

% Abkürzungsverzeichnis TeTEX Version
%\usepackage[german]{nomencl}
%\makenomenclature
%%\makeglossary
%\renewcommand{\nomname}{Abkürzungsverzeichnis}
%\setlength{\nomlabelwidth}{.25\hsize}
%\renewcommand{\nomlabel}[1]{#1 \dotfill}
%\setlength{\nomitemsep}{-\parsep}

% Disable single lines at the start of a paragraph (Schusterjungen)
\clubpenalty = 10000
% Disable single lines at the end of a paragraph (Hurenkinder)
\widowpenalty = 10000
\displaywidowpenalty = 10000

%\usepackage{xcolor}
\usepackage{colortbl}
\usepackage{pifont}
\newcommand*\rot{\rotatebox{90}}
\newcommand*\ok{\cellcolor{green!35}\ding{51}}
\newcommand*\nok{\cellcolor{red!35}\ding{55}}
\usepackage{tablefootnote}
\usepackage{tabularx}
\usepackage{multirow}
\usepackage{capt-of}
\usepackage{enumitem}


\begin{document}
% hier werden die Trennvorschläge inkludiert
\input{latex_einstellungen/trennung}

% Blocksatz Einstellung
\hyphenpenalty=5000
\exhyphenpenalty=5000
\sloppy

%Schriftart Helvetica
%\changefont{phv}{m}{n}

% Leere Seite am Anfang
%\thispagestyle{empty} % erzeugt Seite ohne Kopf- / Fusszeile
%\section*{ }

% Titelseite %
\thispagestyle{empty}

%\begin{figure}[t]
% \includegraphics[width=0.6\textwidth]{abb/fh_koeln_logo}
%\end{figure}

\begin{figure}[t]
 \centering
 \includegraphics[width=0.8\textwidth]{abb/logo}
\end{figure}


\begin{verbatim}


\end{verbatim}

\begin{center}
\Large{Hochschule Bremen}\\
\Large{ZIMT}\\
\end{center}


\begin{center}
\Large{Fakultät für Elektrotechnik und Informatik}
\end{center}
\begin{verbatim}




\end{verbatim}
\begin{center}
\doublespacing
\textbf{\LARGE{\titleDocument}}\\
\singlespacing
\begin{verbatim}

\end{verbatim}
\textbf{{~\subjectDocument~-~Schwerpunkt Technische Informatik}}
\end{center}
\begin{verbatim}

\end{verbatim}
\begin{center}

\end{center}
\begin{verbatim}

\end{verbatim}
\begin{center}
\textbf{zur Erlangung des akademischen Grades \\ Bachelor of Science}
\end{center}
\begin{verbatim}






\end{verbatim}
\begin{flushleft}
\begin{tabular}{llll}
\textbf{Thema:} & & Penetrationstest eines sicherheitskritischen Software-Systems  & \\
& & im Finanzsektor zur Identifizierung von Schwachstellen & \\
&& unter Einsatz des Metasploit-Frameworks & \\
& & \\
\textbf{Autor:} & & Martin Suckau <martin.suckau@web.de> & \\
& & MatNr. 353129 & \\
& & \\
\textbf{Version vom:} & & \today &\\
& & \\
\textbf{1. Betreuer:} & & Prof. Dr. Richard Sethmann &\\
\textbf{2. Betreuer:} & & Prof. Dr. Uwe Meyer &\\
\end{tabular}
\end{flushleft}

% römische Numerierung
%\pagenumbering{arabic}

% 1.5 facher Zeilenabstand
\onehalfspacing

% Sperrvermerk
\thispagestyle{empty}
\section*{Sperrvermerk}
\textcolor{red}{
Die vorliegende Prüfungsarbeit enthält vertrauliche Daten der \domain, die der Geheimhaltung 
unterliegen. Die Prüfungsarbeit wird an der Hochschule Bremen ausschließlich solchen 
Personen zugänglich gemacht, die mit der Abwicklung des Prüfungsverfahrens betraut sind und 
zur Verschwiegenheit verpflichtet sind. Es wird darauf hingewiesen, dass, sofern der 
Verfasser die Bewertung seiner Arbeit angreift, die Arbeit gegebenenfalls dem 
Widerspruchsausschuss zugeleitet werden muss, wobei die Mitglieder des Widerspruchsausschusses zur Verschwiegenheit verpflichtet sind. Wird die Bewertung der Arbeit 
gerichtlich angegriffen, so ist die Arbeit als Teil des Verwaltungsvorgangs dem Gericht zu 
übermitteln. Veröffentlichung und Vervielfältigung der vorliegenden Prüfungsarbeit – auch 
nur auszugsweise und gleich in welcher Form – bedürfen der schriftlichen Genehmigung der \domain.
}

%\thispagestyle{empty} % erzeugt Seite ohne Kopf- / Fusszeile
%\section*{ }

\newpage
\thispagestyle{empty}
\section*{Abstract}
In dieser Bachelor-Thesis...\cite{bsi:studie}

% einfacher Zeilenabstand
\singlespacing

% Inhaltsverzeichnis anzeigen
\newpage
\tableofcontents

\newpage
\addcontentsline{toc}{section}{Abbildungsverzeichnis}
\listoffigures

\newpage
\addcontentsline{toc}{section}{Tabellenverzeichnis}
\listoftables

\newpage
\addcontentsline{toc}{section}{Listingverzeichnis}
\renewcommand{\lstlistlistingname}{Listingverzeichnis}
\lstlistoflistings
%%%%

% das Abkürzungsverzeichnis
%\newpage
% Abkürzungsverzeichnis soll im Inhaltsverzeichnis auftauchen
%\addcontentsline{toc}{section}{Abkürzungsverzeichnis}
% das Abkürzungsverzeichnis entgültige Ausgeben
%\fancyhead[L]{Abkürzungsverzeichnis} %Kopfzeile links

%\printnomenclature

% Definiert Stegbreite bei zweispaltigem Layout
\setlength{\columnsep}{25pt}

%%%%%%% EINLEITUNG %%%%%%%%%%%%
%\twocolumn
\newpage
\fancyhead[L]{\nouppercase{\leftmark}} %Kopfzeile links

% 1,5 facher Zeilenabstand
\onehalfspacing

% einzelne Kapitel
\section{Motivation}\label{einleitung}
Die Commerzbank AG ist laut \textcite[][2]{handelsblatt:commerzbank} Deutschlands zweitgrößtes Finanzinstitut. Als solches stellen sich für ihre IT besondere Herausforderungen an Daten-, Ausfallsicherheit und Stabilität. Regulatorische Vorgaben durch die Bundesanstalt für Finanzdienstleistungsaufsicht\cite{bafin-banken}, interne Programmierrichtlinien\cite{coba-programmierrichtlinienAllgemein, coba-programmierrichtlinienJavaScript}, das Book of Standards \cite{coba-bookOfStandards} und Weitere führen zu einem trägen und wenig innovativem IT-Umfeld. Somit ist es nicht überraschend, dass zur Entwicklung von Webanwendungen noch immer auf alte und etablierte Technologien wie JavaServer Pages und jQuery zurückgegriffen wird\todo{Quelle!}.

Viele Unternehmen nutzen bereits seit einigen Jahren Angular als Technologiebasis für clientseitige Webanwendungen, beispielsweise \textit{Gmail}, \textit{PayPal} oder \textit{Youtube}\todo{quellen}. In der Commerzbank wurde Angular bisher nicht berücksichtigt; erst in jüngerer Vergangenheit wird es in vereinzelten Projekten eingesetzt. Hierbei werden automatisierte Entwicklertests in JavaScript aufgrund von Unwissenheit über die Möglichkeiten meist stiefmütterlich behandelt. Die Anwendungen werden stattdessen per Hand im Webbrowser getestet. Die sich hieraus ergebende Testabdeckung steht im Gegensatz zu den Anforderungen an Sicherheit und Stabilität im Bankenumfeld.

In dieser \titleDocument~ sollen daher die Möglichkeiten zur Realisierung von automatisierten Tests in AngularJS-Webanwendungen untersucht werden. \todo{...?}

\newpage
\section{Grundlagen}\label{grundlagen}
\subsection{Test}
Der Test von Software dient dazu, mögliche Fehler aufzudecken und dadurch die Qualität zu erhöhen. Der Nachweis von Fehlerfreiheit ist unmöglich, daher muss der Testaufwand verhältnismäßig zum Ergebnis sein\cite[][14\psqq]{spillner}. 

Beim Testen werden üblicherweise vier Teststufen unterschieden\cite[][42\psq]{spillner}:
\begin{itemize}
	\item Komponententest
	\item Integrationstest
	\item Systemtest
	\item Abnahmetest
\end{itemize}
Bei Komponenten-, Integrations- sowie bedingt bei Systemstests handelt es sich um Entwicklertests, weshalb diese im Rahmen dieser \titleDocument{} relevant sind. Der Abnahmetest wird daher nicht betrachtet.

\subsubsection{Komponententest}
Ein Komponententest überprüft die einzelnen Bausteine der entwickelten Software erstmalig und unabhängig von anderen Bausteinen. Es wird überprüft, ob die Komponente den Anforderungen sowie dem definierten Softwaredesign entspricht. Außerdem kann auch der Quellcode analysiert und zur Erstellung der Testfälle herangezogen werden; dann handelt es sich um einen \textit{Whitebox-Test}.\cite[][44]{spillner} In JavaScript ist die kleinste testbare Komponente üblicherweise eine Funktion\cite{smashing-unit}. Die zu testende Komponente muss nicht zwingend atomar sein, d.\,h. die Funktion kann aus weiteren Funktionen zusammengesetzt sein, jedoch sollte nur die komponenteninterne Funktionsweise getestet werden\cite[][45]{spillner}. Beim Test von AngularJS-Anwendungen sind die Komponenten beispielsweise Controller, Services, Direktiven und Filter.

Ein Komponententest hat spezifische Testziele. Das Wichtigste ist die Sicherstellung, dass die Funktion der Anforderung in der Spezifikation entspricht. Hierdurch wird die Komponente wie gefordert mit anderen Komponenten zusammenarbeiten und somit in die Gesamtsoftware integriert werden können. Wichtig ist auch der Test auf Robustheit: Bei falschem Aufruf, also einem Verstoß gegen die Vorbedingungen, sollte die Komponente sinnvoll reagieren und den Fehler abfangen. Testfälle lassen sich in Positiv- und Negativtests unterteilen: Positivtests sind die Überprüfung von vorgesehenem Verhalten der Komponente, Negativtests der Test von nicht vorgesehenen, unzulässigen oder explizit ausgeschlossenen Sonderfällen.\cite[][48]{spillner}.

Im Komponententest können auch nicht funktionale Qualitätseigenschaften getestet werden. Zu nennen sind hier beispielsweise Speicherverbrauch oder Antwortzeit, sowie statische Tests auf Wartbarkeit, wie beispielsweise vorhandene Quelltextkommentare oder die Einhaltung von Programmierrichtlinien.\cite[][49\psq]{spillner}.

\subsubsection{Integrationstest}
Der Integrationstest folgt nach den Komponententests und basiert auf getesteten Komponenten. Diese Komponenten werden zu größeren Komponenten oder Teilsystemen zusammengesetzt. Der Integrationstest dient dann dazu zu überprüfen ob alle Einzelteile korrekt zusammenarbeiten und soll Fehler in Schnittstellen und im Zusammenspiel aufdecken. \cite[][52\psq]{spillner} Beispielsweise kann in einem Integrationstest auch die Anbindung an externe Komponenten, wie Datenbanken oder REST-APIs überprüft werden. Diese werden im Komponententest durch \textit{Mocks} ersetzt und emuliert.

Somit ist das aufdecken von Schnittstellenfehlern ein Testziel des Integrationstest, zum Beispiel wegen von der Spezifikation abweichender Schnittstellen. Außerdem ist ein Testziel unerwünschte Wechselwirkungen zwischen den Einzelkomponenten aufzudecken, welche das Zusammenspiel unmöglich machen. \cite[][56]{spillner}

Der Integrationstest ist jedoch kein Ersatz für den Komponententest, da er mit Nachteilen verbunden ist. Es ist schwer bis unmöglich, die tatsächliche Fehlerursache herauszufinden, da oft nicht klar ist in welcher Teilkomponente der Fehler aufgetreten ist, sondern dieser sich nur in einem abweichenden Gesamtverhalten äußert. Manche Fehler werden möglicherweise gar nicht gefunden, da regelmäßig kein vollumfänglicher Zugriff auf Einzelkomponenten besteht.\cite[][57]{spillner}

Es existieren verschiedene Integrationsstrategien, die Auswirkungen auf die Integrationstests haben:\cite[][59\psq]{spillner}
\begin{itemize}
	\item Bei der Top-Down-Integration beginnt der Test mit der obersten Systemkomponente, von der alle Anderen aufgerufen werden. Sukzessive werden die weiteren Komponenten von oben nach unten integriert und getestet, wobei die untergeordneten Komponenten zunächst durch Platzhalter ersetzt werden. Vorteilhaft ist, dass keine aufwändigen Testtreiber zum Aufruf benötigt werden. Jedoch müssen Platzhalterkomponenten implementiert werden, was einen zusätzlichen Overhead beim Test bedeuten kann.
	\item Bei der Bottom-up-Integration werden zunächst die unteren, atomaren Komponenten integriert und getestet. Erst nach und nach werden größere Teilsysteme aus getesteten Komponenten integriert. Der Vorteil hierbei ist, dass keine Platzhalter implementiert werden müssen. Jedoch müssen hier aufwändige Testtreiber erstellt werden, welche die obergeordneten, aufrufenden Komponenten emulieren.
	\item Bei der Ad-hoc-Integration werden Komponenten integriert, sobald sie fertiggestellt sind. Nachteilig hierbei ist, dass sowohl Platzhalter als auch Testtreiber implementiert werden müssen. Allerdings bietet sich ein Zeitgewinn, da jede Komponente so früh wie möglich integriert wird.
	\item Bei der wenig empfehlenswerten Big-Bang-Integration werden alle Komponenten auf einmal integriert. Sie bietet ausschließlich Nachteile: Es wird Zeit verschwendet, da bis zur Fertigstellung der letzten Komponente gewartet wird. Auch treten alle Fehlerwirkungen gesammelt auf, so dass es schwierig ist, die Fehler zu finden.
\end{itemize}

\subsubsection{Systemtest}
Der Systemtest ist der finale Entwicklertest nach den abgeschlossenen Integrationstests. Getestet wird das gesamte System, möglichst in einer produktionsnahen Umgebung. Es sind keine Testtreiber oder Platzhalter mehr vorhanden, stattdessen wird überall die finale Hard- und Software genutzt. Das Testziel ist die Validierung, ob alle funktionalen und nicht-funktionalen Anforderungen erfüllt werden.\cite[][60\psqq]{spillner}

Aufgrund der erforderlichen Produktionsnähe sowie der erforderlichen Datenbasis kann der Systemtest nur bedingt als Entwicklertest gesehen werden\cites[236]{roitzsch}[]{oose}. Jedoch kann und sollte ein einfacher Systemtest auch von Entwicklern durchgeführt werden. Es bietet sich hier die Durchführung von End-To-End-Tests an, mittels derer das System von der Benutzeroberfläche bis zur untersten Komponentenschicht getestet werden kann\cite{softwaresanierung}.

\subsection{AngularJS}
AngularJS ist ein von Google ins Leben gerufenes JavaScript-Framework zur Entwicklung von clientseitigen Webanwendungen \cite{angular-faq}. Der Quellcode von AngularJS steht auf Github zur Verfügung und wird dort von einer großen Entwicklergemeinschaft weiterentwickelt\cite[][9]{ng-book}. Da es unter der MIT-Lizenz veröffentlicht ist eignet sich AngularJS auch für den kommerziellen Einsatz\cites[9]{ng-book}[]{mit-license}.

Ende 2016 wurde eine neue Version von Angular veröffentlicht: Angular2 \cite{ng2-out}. Durch die Bezeichnung kann AngularJS (Version 1) klar von Angular2 (Version 2) abgegrenzt werden. Im Rahmen dieser Bachelorarbeit wird AngularJS betrachtet.

\subsubsection{Architektur}
Bei der Entwicklung von Webanwendungen mit AngularJS kommt das Model-View-ViewModel-Entwurfsmuster (MVVM), eine Erweiterung von Model-View-Controller (MVC), zum Einsatz\cite[][21]{angular-boehm}. 

Die Model-Schicht, also die Datenhaltung und Geschäftslogik, liegt hierbei auf dem Server und wird durch REST- oder WebSocket-Verbindungen dargestellt. Hierzu kommen in AngularJS meist Services zum Einsatz: Vordefinierte, wie z.\,B. der http-Service für HTTP-Abfragen, oder Selbsterstellte \cite{angular-services}. Mittels dieser kann Geschäftslogik auch clientseitig umgesetzt werden.\cite[][21]{angular-boehm}

Es ist erforderlich, die über die Model-Schicht ermittelten Daten zu verwalten und gegebenenfalls zu transformieren um sie der Anzeige zur Verfügung zu stellen. Hierfür wird die ViewModel-Schicht genutzt. Außerdem wird in dieser Schicht die Funktionalität definiert, welche die View-Schicht steuert und diese zur Kommunikation mit der Model-Schicht nutzt. Dabei handelt es sich um Funktionen zur Behandlung von Events, wie Buttonclicks, Texteingaben, etc. Zur Weitergabe der Daten an die Anzeige wird Two-Way-Databinding (s. Abschnitt \ref{sec:twoWayDatabinding}) verwendet. Umgesetzt wird die ViewModel-Schicht mit Controllern und sogenannten Scopes (s. Abschnitt \ref{sec:scopes}). \cite[][21\psq]{angular-boehm}

Die View-Schicht wird in AngularJS mit Templates und Direktiven umgesetzt. Templates sind HTML-Dateien, in welchen zusätzliche Tags und Attribute, die sogenannten Direktiven, verwendet werden. \cite[][1\psqq]{angular-boehm}Direktiven ermöglichen es wiederverwendbare Komponenten zu erschaffen, indem Template und Quelltext in einem neuen Tag oder Attribut gekapselt werden\cite[][49\psq]{angular-boehm}.

\begin{figure}[H]
\lstinputlisting[caption={Beispiel eines AngularJS-Templates, adaptiert nach  \cite{angular-boehm}}, label=lst:angularTemplate]{lst/angularTemplate.html}
\end{figure}

Im Beispieltemplate (s. Listing \ref{lst:angularTemplate}) wird ein Eingabefeld (HTML \texttt{input}) definiert, dessen Inhalt automatisch mit der im Scope liegenden Variable \texttt{someModelField} synchronisiert wird. Ein \texttt{h1}-Element zeigt den Inhalt dieser Variablen an. Für beide Synchronisierungen wird automatisch Two-Way-Databinding (s. Abschnitt \ref{sec:twoWayDatabinding}) genutzt. Weiterhin wird ein Button definiert, welcher bei Click die Controller-Funktion \texttt{setName()} aufrufen soll. Die Angabe \texttt{ng-app} im \texttt{html}-Tag gibt das AngularJS-Modul an, welches von der Anwendung verwendet werden soll. Die Angabe \texttt{ng-controller} spezifiziert den von diesem Template zu verwendenden Controller (s. Listing \ref{lst:angularTestCtrl}).

\begin{figure}[H]
	\lstinputlisting[caption={Beispielhafter AngularJS-Controller, adaptiert nach   \cite{angular-controller}}, label=lst:angularTestCtrl]{lst/angularTestCtrl.js}
\end{figure}

In der JavaScript-Datei wird im verwendeten Modul eine Funktion, die als Controller mit dem Namen \texttt{TestCtrl} dient, definiert. Dieser Controller spezifiziert die Funktion \texttt{setName}, welche dadurch im Template verwendet werden kann. Das Skript muss über Dateikonkatenation (npm-Package \glqq concat\grqq  \cite{npm-concat}) oder zusätzliches Einbinden in das Template an den Browser ausgeliefert werden.

\subsubsection{Two-Way-Databinding}
\label{sec:twoWayDatabinding}
Two-Way-Databinding ist die Datenbindung in beide Richtungen. Es dient der Aktualisierung der Model-Daten anhand von Benutzereingaben in der Ansicht sowie die Anpassung und Aktualisierung der View bei Änderungen des zugrundeliegenden Datenmodells. Dieses Konzept ist integraler Bestandteil von AngularJS und erspart das Schreiben von Boilerplate-Code, der nicht zur Geschäftslogik beiträgt. Ohne Two-Way-Databinding wäre es erforderlich, auf jedem zu synchronisierenden DOM-Element einen ChangeListener zu registrieren, welcher Änderungen durch den Benutzer an das Datenmodell weiterreicht. Außerdem müsste Logik implementiert werden, welche bei einer Änderung von Variablen im Datenmodell die View aktualisiert. Die Datenbindung in AngularJS erhöht somit die Effizienz, da Programmcode mit weniger Overhead geschrieben werden kann.\cite[][24]{angular-boehm}

\subsubsection{Scopes}
\label{sec:scopes}
Scopes sind in AngularJS die Basis der Datenbindung, wobei in einem Scope die Variablen und Funktionen definiert sind, welche für einen bestimmten Teil des DOM benötigt werden. Scopes sind hierarchisch angeordnet und bilden grob die DOM-Struktur nach. Den Ursprung dieser Hierarchie bildet der Root-Scope, welcher von AngularJS standardmäßig zur Verfügung gestellt wird. Hierbei können sie entweder die Eigenschaften des jeweils übergeordneten Scopes erben oder isoliert sein. Beim Auswerten von Ausdrücken in Templates (z.\,B. \texttt{\{\{scopeVariable\}\}}) wird zuerst im mit dem jeweiligen Element assoziierten Scope und danach in den jeweils Übergeordneten nach der Eigenschaft gesucht.\cites[23\psqq]{angular-boehm}[]{angular-scopes}

Zur Erkennung, ob eine Variable im Datenmodell geändert wurde und eine Aktualisierung der Anzeige erforderlich ist, wird in AngularJS Dirty Checking genutzt. Hierzu wird von jedem Scope eine Kopie im Speicher gehalten, so dass bei jedem Event die gehaltene und aktuelle Version eines Scopes miteinander verglichen werden können. Bei veränderten Werten wird eine Aktualisierung der Anzeige angestoßen.\cites[24]{angular-boehm}[]{angular-dirty}

\subsubsection{Dependency Injection}
Dependency Injection ist ein Entwurfsmuster welches beschreibt, wie eine Komponente Zugriff auf benötigte Abhängigkeiten, also andere Komponenten, bekommt und wird in AngularJS durchgängig genutzt. Bei Nutzung von Dependency Injection werden die Komponenten nicht selber erzeugt sondern von außerhalb durch einen Injector geliefert. Hierfür ist es nötig, dass Services, Direktiven, Filter und Controller mit den entsprechenden Factory-Funktionen von AngularJS erzeugt werden. Diese registrieren einerseits die Komponente und ermöglichen es andererseits, diese in andere Komponenten zu injizieren. Des Weiteren kümmern sie sich aber auch um die Bereitstellung der benötigten Komponenten.  \cite{angular-di}

\begin{figure}
	\lstinputlisting[caption={Beispielhafter AngularJS-Controller mit Dependency Injection, adaptiert nach  \cite{angular-di}}, label=lst:angularTestCtrlDI]{lst/angularTestCtrlDI.js}
\end{figure}

Im Beispiel in Listing \ref{lst:angularTestCtrlDI} wird im Modul \texttt{someModule} ein neuer Controller \texttt{MyController} angelegt, in welchen der Scope, sowie die zwei Services \texttt{http}, welcher standardmäßig von AngularJS zur Verfügung gestellt wird, sowie \texttt{dep}, welcher benutzerdefiniert erstellt wurde, injiziert werden. Diese Abhängigkeiten werden durch Übergabe als Funktionsparameter bereitgestellt.

Dependency Injection bietet gravierende Vorteile für die Testbarkeit. Es ermöglicht, eine Komponente durch ein spezielles selbst implementiertes Mock-Objekt zu ersetzen, dessen Verhalten festgelegt werden kann. Bei Tests kann das Verhalten der Abhängigkeiten festgelegt und Komponenten isoliert getestet werden.\cite[][27]{angular-boehm}

\subsection{Node.js}
\subsubsection{Laufzeitumgebung}
Node.js ist eine Laufzeitumgebung, mit der JavaScript ohne Webbrowser ausgeführt werden kann\cite[][1]{hughes2012einfuehrungnodejs}. Somit ist es möglich, JavaScript nicht nur für die Darstellung von Benutzeroberflächen im Webbrowser zu Nutzen, sondern auch als Backend-Sprache oder zur Unterstützung von Entwicklungsprozessen auf Continuous-Integration-Servern oder Entwicklerarbeitsplätzen.

Intern nutzt Node.js die JavaScript-Engine \textit{Chrome V8} \cite{nodejs}, welche von Google als Open-Source-Software veröffentlicht wurde. V8 kommt auch im weitverbreiteten Webbrowser \textit{Google Chrome} zum Einsatz und implementiert den JavaScript-Standard ECMAScript wie in ECMA-262 spezifiziert\cite{chromev8}. ECMA-262 ist der im Juni 2016 veröffentlichte und zurzeit aktuellste JavaScript-Standard\cite{ecma262}. Somit bietet Node.js alle spezifizierten und von Google Chrome unterstützten Sprachfunktionalitäten. Es eignet sich daher auch für den Test von für Webbrowser entwickelte Webanwendungen.

\subsubsection{npm}
Der Node Package Manager (npm) ist der zusammen mit Node.js installierte Paketmanager für JavaScript. Unter npm wird außerdem die \textit{npm Registry}, also die zentrale Ablage von mit npm verwendeten JavaScript-Paketen, verstanden, auf welche der Node Package Manager zugreift. \cite{npm-about} Die npm Registry enthält über 180.000 Pakete und ist damit das größte öffentliche Softwarerepository\cite{modulecount}.

Grundlegend funktioniert die Paketverwaltung mit einer JSON-Konfigurationsdatei, der \texttt{package.json} (vgl. Listing \ref{lst:package-json}). Die Datei enthält den Namen sowie die Version des Pakets, für welches sie angelegt wurde, sowie optional weitere Metadaten wie Beschreibung, Autor und Referenzlinks auf Bugtracker. Außerdem werden hier Abhängigkeiten angegeben, die zur Ausführung (\texttt{dependencies}) oder zur Entwicklung (\texttt{devDependencies}) in diesem Paket benötigt werden.\cite{npm-packagejson} Die angegebenen Abhängigkeiten werden von npm automatisiert heruntergeladen und im Ordner \texttt{node\_modules} abgelegt, von wo aus sie in die JavaScript-Anwendung eingebunden werden können. Auch transitive Abhängigkeiten werden von npm aufgelöst.\cite{npm-install}

\begin{figure}[H]
	\lstinputlisting[caption=Beispiel einer package.json, label=lst:package-json]{lst/package.json}
\end{figure}

Neben der Paketverwaltung kann npm auch zum Build als Taskrunner eingesetzt werden. Hiermit kann der Buildprozess eines Paketes automatisiert werden, z.\,B. durch die automatisierte Ausführung von Tests oder dem Aufrufen von Compilern. Hierzu werden in der \texttt{package.json} Skripte angegeben. Diese bestehen aus einem Skript-Namen und dem auszuführenden Befehl. Im Beispiel (siehe Listing \ref{lst:packageWithScripts-json}) werden drei Skripte definiert:\cite{cirkel-npmAsABuildTool}
\begin{itemize}
	\item \glqq lint\grqq~ führt das Kommando \texttt{jshint **.js} aus. Dies dient dem Überprüfen von JavaScript-Dateien auf statische Programmierfehler\cite{jshint-about}.
	\item \glqq build\grqq~ führt das Kommando \texttt{browserify [...]} aus. Dieses dient dem Zusammenfügen von mittels \texttt{require} eingebundenen JavaScript-Dateien in eine konkatenierte Datei\cite{browserify-about}.
	\item \glqq test\grqq~ führt das Kommando \texttt{mocha [...]} aus. Mocha ist ein Test-Runner (siehe auch Abschnitt \ref{sec:Mocha}).
\end{itemize}

Angegebene Skripte können auch automatisch in sogenannten Hooks (\textit{Pre} und \textit{Post Hooks}) ausgeführt werden. Im Beispiel (siehe Listing \ref{lst:packageWithScripts-json}) sind folgende Hooks definiert:\cite{cirkel-npmAsABuildTool}
\begin{itemize}
	\item \glqq prepublish\grqq{} wird vor der Ausführung von \texttt{publish}, welches ein npm Standard-Skript ist und das Paket in der npm Registry veröffentlicht\cite{npm-publish}, das benutzerdefinierte Skript \texttt{build} sowie dadurch \texttt{prebuild} und \texttt{postbuild} ausführen.
	\item \glqq prebuild\grqq{} wird vor Ausführung des \texttt{build}-Skripts das Paket durch Ausführung von \texttt{test} überprüfen.
	\item \glqq pretest\grqq{} wird vor Ausführung von \texttt{test} mittels \texttt{lint} das Paket auf statische Fehler untersuchen.
\end{itemize}

\begin{figure}[H]
	\lstinputlisting[caption={Beispiel von Skripten in einer package.json (aus \cite{cirkel-npmAsABuildTool})}, label=lst:packageWithScripts-json]{lst/packageWithScripts.json}
\end{figure}

Die wohl populärste Funktion von Taskrunnern ist das automatisierte Beobachten des Dateisystems auf Änderungen. Häufig ist es wünschenswert, dass bei einer Dateiänderung automatisch ein entsprechender Buildprozess oder die Tests ausgeführt werden. Diese Funktionalität bietet npm im Gegensatz zu anderen Taskrunnern wie \textit{Gulp} oder \textit{Grunt} nicht nativ, sondern nur mithilfe eines Zusatzpakets. Ein entsprechendes Beispiel, welches bei Veränderung einer Datei im Paket-Ordner die JavaScript-Module zu einer Datei konkateniert\cite{browserify-about} und den JavaScript-Code des Paketes überprüft, findet sich in Listing \ref{lst:packageWithWatch-json}.\cite{cirkel-npmAsABuildTool}

\begin{figure}[H]
	\lstinputlisting[caption={Beispiel der Watch-Funktionalität in einer package.json (aus \cite{cirkel-npmAsABuildTool}, angepasst durch den Autor)}, label=lst:packageWithWatch-json]{lst/packageWithWatch.json}
\end{figure}

\newpage
%\section{Systeminformationen}\label{system_info}
\section{Auswahl von Testsoftware}
Im folgenden Kapitel wird eine Auswahl verschiedener Tools und Frameworks zur Realisierung von automatisierten Tests anhand von vorher zu definierenden Anforderungen getroffen.
\subsection{Anforderungsanalyse}
\label{sec:anforderungsanalyse}
An die auszuwählenden Frameworks oder Tools beziehungsweise eine Kombination Mehrerer stellen sich die folgenden Anforderungen:
\begin{enumerate}[label=\textbf{A\arabic*}]
	\item Es muss die Durchführung von Komponententests möglich sein.
	\item Es muss die Durchführung von End-To-End-Tests möglich sein.
	\item Es muss der Test von AngularJS-Anwendungen möglich sein.
	\item Es muss die Testausführung von Komponententests im Browser aus der Konsole heraus möglich sein.
	\item Sie muss Open-Source und für den kommerziellen Einsatz freigegeben sein.
	\item Sie soll durch eine Open-Source-Community aktiv gewartet werden; Fehler in ihr sollen behoben werden.
	%\item Es muss mindestens der aktuelle Commerzbank-Standard, ECMA-262 5.1, unterstützt werden\cite[][10]{coba-programmierrichtlinienJavaScript}.
	\item Der Testcode soll leicht lesbar sein\cite[][7]{coba-programmierrichtlinienAllgemein}; hierzu bietet sich der BDD-Stil an, den das Testframework somit unterstützen soll.
	\item Die Software soll eine geringe Einarbeitungszeit erfordern und problemlos zu verwenden sein. 
	\item Sie soll dem Entwickler Flexibilität beim Schreibe der Testfallimplementationen und Assertions bieten, damit dieser möglichst nah an seinen persönlichen Präferenzen arbeiten kann.
	\item Die Ausgabe der Testergebnisse muss konfigurierbar sein, um sie gegebenenfalls in Prozessen oder anderen Tools weiterverwenden zu können.
	\item Bei der Durchführung von End-To-End-Tests soll es möglich sein, die Ergebnisse mit Screenshots zu dokumentieren.
	\item Der Einsatz von Spies, Stubs und Mocks muss bei Unit-Tests möglich sein.
\end{enumerate}

\newpage
Aufgrund diverser regulatorischer Vorgaben, Prozesse und vorhandener technischer Systeme innerhalb des Commerzbank-Konzerns ergeben sich weitere Anforderungen:
\begin{enumerate}[label=\textbf{CB\arabic*}]
	\item Die Software muss unter Windows 7 funktionieren.
	\item Es darf keine zusätzliche Software zur Ausführung erforderlich sein; lediglich Node.js und Java sind zulässig, da diese über die Standardsoftwareverteilung der Commerzbank verfügbar sind.
	\item Wenn die Software Node.js benötigt, muss sie mit Version 6.9.2 lauffähig sein.
	\item Die Software muss über den Node Package Manager installierbar sein.
	\item Gemäß dem SEF der Commerzbank müssen Testfälle eindeutig beschrieben werden; diese Beschreibung auch kann durch Implementierung der Tests erfolgen\cite{coba-sef}. Das Testframework soll daher eine Beschreibung der Tests forcieren.
	\item Gemäß dem SEF sollen Komponententests \glqq einen möglichst großen Teil der Funktionalität abdecken\grqq{}\cite{coba-sef}. Zur Überprüfung dessen muss eine Möglichkeit zur Ermittlung der Code Coverage bestehen.
	\item Das Bearbeiten aller Dateien in der TFS-Quellcodeverwaltung soll möglich sein, es dürfen somit keine proprietären Binärdateien zum Einsatz kommen.
\end{enumerate}

Zur Erfüllung der Anforderung A1 werden ein \textit{Testrunner}, ein \textit{Testframework} zum Anlegen und Beschreiben der Testfälle sowie eine \textit{Assertion-Bibliothek} benötigt. Zur Erfüllung von Anforderung A2 wird zusätzlich ein \textit{Tool zur Browsersteuerung} benötigt.

\newpage
\subsection{Softwarerecherche}
Einen Überblick über zur Verfügung stehende Tools bieten \cite{unittest-overview} und Weitere.

\subsubsection{Karma}
\label{sec:Karma}
Karma ist ein Test-Runner für die Ausführung von JavaScript-Tests. Er wurde vom AngularJS-Team ins Leben gerufen und wird auf GitHub von einer Open-Source-Gemeinschaft weiterentwickelt.\cite{karma-index} Karma liegt als Paket \texttt{karma} im npm-Repository\cite{karma-faq}.

Karma ermöglicht die Nutzung diverser Testframeworks, wie Jasmine (s. Abschnitt \ref{sec:Jasmine}), Mocha (s. Abschnitt \ref{sec:Mocha}) oder QUnit (s. Abschnitt \ref{sec:QUnit}). Auch Continuous Integration Server wie Jenkins oder Travis werden unterstützt.\cite{karma-faq}

Grundlegend basiert Karma auf einem Client-Server-Prinzip, wobei Karma einen Webserver startet, welcher alle verbundenen Browser fernsteuert und in diesen die Tests ausführt. Ein Browser kann hierbei entweder manuell, also durch Aufruf der vom Karma-Server bereitgestellten URL, oder automatisiert, also indem der Browser durch Karma gestartet wird (vgl. Listing \ref{lst:karma-conf-js}), verbunden werden. In jeder Testumgebung wird der Quelltext mittels IFrame eingebunden, der Test ausgeführt und danach die Ergebnisse an den Server gesendet. Dort werden die Ergebnisse aufgearbeitet präsentiert oder automatisiert von übergeordneten Buildprozessen verarbeitet. Das verwendete Prinzip stammt aus einer Masterthesis \cite{karma-masterThesis}; tieferes Verständnis ist jedoch für die reine Nutzung von Karma nicht erforderlich.\cite{karma-howItWorks}

Für die Konfiguration wird eine JavaScript-Datei, die \texttt{karma.conf.js}, genutzt. Ein Beispiel findet sich in Listing \ref{lst:karma-conf-js}. Mit dieser beispielhaften Konfiguration wird für die Testausführung das Framework Jasmine (s. Abschnitt \ref{sec:Jasmine}) genutzt. Der Parameter \texttt{files} gibt an, welche Dateien von Karma ausgeliefert und beobachtet werden, und somit bei Änderung welcher Dateien die Tests automatisch erneut ausgeführt werden. Außerdem ist konfiguriert, dass bei Testdurchführung automatisch Firefox gestartet werden und in diesem die Tests ausgeführt werden soll.\cite{karma-configurationFile, karma-files}

\begin{figure}[H]
	\lstinputlisting[caption={Beispiel einer \texttt{karma.conf.js} (adaptiert nach \cite{karma-configurationFile, karma-files})}, label=lst:karma-conf-js]{lst/karma.conf.js}
\end{figure}

Die Funktionalität von Karma lässt sich mit Plugins erweitern. Sie werden für die Einbindung von Testframeworks, ein verändertes Ausgabeformat der Testergebnisse, Präprozessoren (z.\,B. für die Auslieferung von in JavaScript eingebettetem HTML oder die Ermittlung der Code Coverage)\cite{karma-preprocessors} oder die Einbindung von Browsern wie Firefox, Chrome oder PhantomJS (s. Abschnitt \ref{sec:PhantomJS}) benötigt. Jedes Plugin ist ein npm-Paket, daher werden Plugins über npm installiert. Karma bindet alle installierten Pakete mit dem Namen \texttt{karma-*} automatisch ein.\cite{karma-plugins} Im npm-Repository liegen über 1300 Karma-Plugins.\cite{karma-npm}

\subsubsection{Mocha}
\label{sec:Mocha}
Mocha ist ein Testframework und Test-Runner, welches sowohl in Node.js als auch in Browsern lauffähig ist. Es liegt als Paket \texttt{mocha} im npm-Repository und kann darüber installiert werden.\cite{mocha-index}

Tests bestehen in Mocha aus drei Ebenen. Die Oberste sind die Testsuites, welche weitere Testsuites oder Testfälle enthalten können. Testfälle bestehen aus funktionalem Code sowie Assertions als eigentliche Testüberprüfung. Es werden verschiedene Stile zur Testbeschreibung unterstützt: BDD, TDD, QUnit und weitere, welche sich nur in ihrem Aussehen unterscheiden und Entwicklern ermöglichen, ihren eigenen Stil zur Definition von Tests zu wählen.\cite{mocha-index}

Für Assertions können in Mocha verschiedene Frameworks genutzt werden. In \cite{mocha-index} wird beispielsweise die Nutzung von \textit{should.js} bei Verwendung des BDD-Stils, \textit{expect.js} oder \textit{chai} (s. Abschnitt \ref{sec:Chai}) empfohlen. Es ist einem Entwickler somit möglich, Mocha auf die eigenen Vorlieben anzupassen.\cite{mocha-index} Auf eine genauere Vorstellung und Codebeispiele wird an dieser Stelle aufgrund der Vielseitigkeit verzichtet.

Mocha ermöglicht es, die Testausgabe beliebig zu konfigurieren, so dass beispielsweise eine Ausgabe in der Konsole, als HTML-Datei, als JSON oder im XML-Format möglich ist. Bei besonderen Anforderungen können eigene Reporter erstellt werden, z.\,B. zur Einbindung in Continuous-Integration-Tools.

\subsubsection{AVA}
\label{sec:Ava}
AVA ist ein Test-Runner für in JavaScript geschriebene Tests. Die Besonderheit ist, dass die Tests simultan ausgeführt werden und somit deutliche Performanceverbesserungen möglich sind. Es steht unter \texttt{ava} im npm-Repository zur Verfügung.\cite{ava}

AVA läuft ausschließlich in Node.js; es gibt also keine Möglichkeit den Test-Runner im Browser aufzurufen. Ein Test wird über Aufruf einer Funktion, welche aus dem Node-Modul \textit{ava} importiert wird, definiert. Dieser Funktion wird ein Funktionskörper übergeben, welcher den Test spezifiert und Assertions durchführt. Ein Beispiel findet sich in Listing \ref{lst:ava}. Die Assertion-Funktionen werden über das übergebene Testobjekt \texttt{t} bereitgestellt. Es ist nicht vorgesehen andere Assertionframeworks zu nutzen. Mit den Funktionen \texttt{before}, \texttt{after}, \texttt{beforeEach} und \texttt{afterEach} können Funktionen definiert werden, welche vor oder nach jedem Test oder einmalig allen Tests ausgeführt werden.\cite{ava}

\begin{figure}[H]
	\lstinputlisting[caption={Beispiel eines Tests mit AVA (adaptiert nach \cite{ava})}, label=lst:ava]{lst/ava.js}
\end{figure}

\subsubsection{QUnit}
\label{sec:QUnit}
QUnit ist ein Framework für automatisierte Komponententests mit JavaScript und wird von jQuery und einer Vielzahl weiterer Projekten genutzt. Es liegt unter \texttt{qunitjs} als Paket im npm-Repository. Es kann sowohl in Browsern als auch in Node.js ausgeführt werden.\cite{qunit-index}

In QUnit geschriebene Tests ähneln denen vieler Testframeworks populärer anderer Sprachen, wie beispielsweise JUnit in Java. Ein Testfall wird mittels Aufruf von \texttt{QUnit.test(string, function)} definiert. In der übergebenen Funktion kann Testcode aufgerufen werden und das Ergebnis mit Assertions validiert werden. Wenn mindestens eine Assertion fehlschlägt, gilt der Test als fehlgeschlagen; sonst als bestanden. QUnit liefert Assertions mit: beispielsweise \texttt{assert.ok}, welche einen truthy Wert erwartet, oder \texttt{assert.equal}, welches zwei als gleich angesehene Werte erwartet.\cite{qunit-cookbook}

Tests können in durch Aufruf von \texttt{QUnit.module(string)} erzeugten Modulen gruppiert werden (s. Listing \ref{lst:qunit-test}). In Modulen kann Code ausgelagert werden, indem die vor und nach jedem Test aufgerufenen Funktionen \texttt{beforeEach} und \texttt{afterEach} definiert werden.\cite{qunit-cookbook}

\begin{figure}[H]
	\lstinputlisting[caption={Beispiel mehrerer Tests für QUnit (adaptiert nach \cite{qunit-cookbook})}, label=lst:qunit-test]{lst/qunit-1.js}
\end{figure}

\subsubsection{Intern}
\label{sec:Intern}
Intern ist ein Framework für automatisierte Tests in JavaScript. Es bietet sowohl Möglichkeiten für Komponenten- als auch für End-To-End-Tests. Intern ist flexibel und bietet dem Entwickler Möglichkeiten, seinen eigenen Stil zu verfolgen. Es steht als Paket \texttt{intern} über npm zur Verfügung.\cite{intern-userguide}

Intern stellt verschiedene Interfaces zur Definierung von Tests bereit; es können auch eigene definiert werden. Das \textit{Object}-Interface ist eine einzelne Funktion, welcher ein Objekt übergeben wird, welches alle definierten Tests enthält. In diesem Objekt werden außerdem Funktionen die vor oder nach jedem Test oder der Testsuite ausgeführt werden sollen definiert. Ein Beispiel findet sich in Listing \ref{lst:intern-object}.\cite{intern-userguide}

\begin{figure}[H]
	\lstinputlisting[caption={Beispiel einer Testsuite in Intern (aus \cite{intern-userguide})}, label=lst:intern-object]{lst/intern-object.js}
\end{figure}

Die Interfaces \textit{BDD} und \textit{TDD} ähneln einander und unterscheiden sich nur durch die Benennung einzelner Funktionen. Sie verfolgen gegenüber dem Object-Interface einen prozeduraleren Ansatz, basieren also auf verschachtelt aufgerufenen Funktionen statt auf Objekten. Auch hier können Funktionen definiert werden, welche vor oder nach jedem Test oder der Suite aufgerufen werden. Ein Beispiel ist in Listing \ref{lst:intern-bdd} zu finden.\cite{intern-userguide}

\begin{figure}[H]
	\lstinputlisting[caption={Beispiel einer Testsuite in Intern (aus \cite{intern-userguide})}, label=lst:intern-bdd]{lst/intern-bdd.js}
\end{figure}

Es steht ein Interface zur Verfügung, welches QUnit nachempfindet und somit die Verwendung von in QUnit geschriebenen Tests (s. Listing \ref{lst:qunit-test} in Abschnitt \ref{sec:QUnit}) in Intern ermöglicht. Für alle Tests gilt, dass ein Test fehlschlägt, wenn in ihm ein Error auftritt, also eine Assertion fehlschlägt. Ansonsten gilt er als bestanden. Die Chai-Bibliothek ist in Intern enthalten, es ist jedoch auch die Nutzung von beliebigen anderen Assertion-Frameworks möglich. Durch Aufruf der Funktion \texttt{skip} können Tests übersprungen werden.\cite{intern-userguide}

End-To-End-Tests werden genau wie Unit-Tests definiert, werden jedoch in der Konfiguration in \texttt{functionalSuites} und nicht in \texttt{suites} aufgeführt. Hierdurch werden sie im Kontext des Test-Runners und nicht in der zu testenden Umgebung ausgeführt. Für die Interaktion mit dem Browser verwendet Intern mit \textit{leadfoot} einen Wrapper für Selenium. Über das in \texttt{this.remote} zur Verfügung gestellte Leadfoot-Command-Objekt können Befehle im Browser ausgeführt werden.\cite{intern-userguide}

\begin{figure}[H]
	\lstinputlisting[caption={Beispiel eines End-To-End-Tests in Intern (aus \cite{intern-userguide})}, label=lst:intern-e2e]{lst/intern-e2e.js}
\end{figure}

Im Beispiel in Listing \ref{lst:intern-e2e} wird ein Testfall \glqq greeting form\grqq{} definiert, in welchem die Index-Seite geladen wird und auf dieser in einem Eingabefeld der Wert \glqq Elaine\grqq{} eingegeben und der Submit-Button geklickt wird. Abschließend wird überprüft ob das Element mit der ID \glqq greeting\grqq{} den korrekten Inhalt enthält.

Intern ermittelt standardmäßig beim Ausführen von Tests die Code Coverage, also die Abdeckung von Codezeilen, Funktionen, Zweigen und Anweisungen. Die Ausgabe sowohl von Code Coverage als auch der Testergebnisse ist konfigurierbar.\cite{intern-userguide}


\subsubsection{Jasmine}
\label{sec:Jasmine}
Jasmine ist ein Behavior Driven Development Framework zum Test von JavaScript\cite{jasmine-introduction}. Es liegt unter \texttt{jasmine} im npm-Repository\cite{jasmine-getting-started}. Jasmine bietet eine saubere und einfache Syntax zur Beschreibung von Testfällen. Die Tests bestehen aus drei Ebenen: Testsuites, Spezifizierungen (\glqq Specs\grqq) und Erwartungen, also den eigentlichen Testassertions\cite{jasmine-introduction}. Ein beispielhafter Test findet sich in Listing \ref{lst:jasmine-1}.

Eine Testsuite beginnt auf oberster Ebene mit dem Aufruf der globalen JavaScript-Funktion \texttt{describe(string, function)}. Der String ist hierbei der Name der Testsuite, üblicherweise wird hier das Testsubjekt benannt. Die Funktion implementiert die Testsuite und besteht aus Specs.\cite{jasmine-introduction}

Ein Spec wird durch Aufruf der globalen Funktion \texttt{it(string, function)} angelegt. Der String enthält eine Beschreibung des Testfalls; nach dem BDD-Modell also eine Beschreibung des erwarteten Verhaltens. Die Funktion dient zum Überprüfen dieses Verhaltens und enthält Assertions, welche entweder \texttt{true} oder \texttt{false} ergeben. Liefern alle Assertions \texttt{true} so gilt die Spec als bestanden, ansonsten als durchgefallen.\cite{jasmine-introduction}

Eine Assertion besteht in Jasmine aus der Funktion \texttt{expect(object)}, welcher der tatsächliche Wert übergeben wird. Diese wird mit einer Matcher-Funktion verkettet, welche den erwarteten Wert übergeben bekommt und die beiden Werte vergleicht und auswertet. Es wird eine Vielzahl an vorgefertigten Matchern mitgeliefert: \texttt{toEqual}, \texttt{toContain}, \texttt{toBeTruthy}, und Weitere\cite{jasmine-introduction, jasmine-cheatsheet}.

\begin{figure}[H]
	\lstinputlisting[caption={Beispiel einer Jasmine-Testsuite (adaptiert nach \cite{jasmine-introduction})}, label=lst:jasmine-1]{lst/jasmine-1.js}
\end{figure}

Specs können als \textit{pending} deklariert werden. Sie werden dann nicht ausgeführt, aber im Ergebnis angezeigt. Hierfür kann beim Aufruf der \texttt{it}-Funktion die Übergabe einer Funktion weggelassen werden, stattdessen die \texttt{xit}-Funktion aufgerufen werden oder im Funktionskörper die \texttt{pending}-Funktion genutzt werden.\cite{jasmine-introduction}

\subsubsection{Chai}
\label{sec:Chai}
Chai ist eine Assertion-Bibliothek, welche mit jedem Testframework kombiniert werden kann. Chai bietet verschiedene Assertion-Stile, so dass der Entwickler seinen eigenen Stil wählen kann.\cite{chai-index} Chai ist unter der ID \texttt{chai} über npm verfügbar und kann so installiert und in Projekte eingebunden werden\cite{chai-installation}.

Der Assert-Stil ähnelt klassischeren Testframeworks wie QUnit oder dem Assert-Modul in Node.JS. Über das \texttt{assert}-Objekt werden Funktionen zur Verfügung gestellt, zum Beispiel \texttt{isOk} zur Überprüfung, ob der Parameter truthy ist oder \texttt{equal} zur Überprüfung, ob die Parameter gleich sind. Jeder Funktion kann eine optionale Nachricht übergeben werden, welche im Falle eines Fehlschlags in der Fehlermeldung angezeigt wird.\cite{chai-assert}

Der BDD-Stil wird in zwei Varianten zur Verfügung gestellt: \texttt{expect} und \texttt{should}. Er ermöglicht es Assertions in einer natursprachlichen Form zu schreiben, welche somit gut lesbar sind. Die should-Variante hat einige Nachteile, weshalb an dieser Stelle nur die expect-Variante betrachtet wird. \texttt{Expect} ist eine Funktion, welcher der zu überprüfende Wert übergeben wird. Diese Methode wird mit weiteren Objekten und Funktionen verkettet, um die Assertion zu bilden.\cite{chai-styles} Ein Beispiel hierzu findet sich in Listing \ref{lst:chai-expect}.

\begin{figure}[H]
	\lstinputlisting[caption={Beispiel von Assertions mit dem expect-Stil von Chai (aus \cite{chai-styles})}, label=lst:chai-expect]{lst/chai-expect.js}
\end{figure}

\subsubsection{Protractor}
\label{sec:Protractor}
Protractor ist ein speziell für Angular-Anwendungen entwickeltes Framework für End-to-End-Tests. Die Tests werden in Browsern direkt gegen die Anwendungsoberfläche durchgeführt und simulieren somit das Verhalten eines echten Benutzers. Es liegt im npm-Repository mit der ID \texttt{protractor} und ist dadurch einfach zu installieren.\cite{protractor-index}

Für die Steuerung des Browsers greift Protractor auf Selenium zurück\cite{protractor-index}, welches den W3C WebDriver-Standard implementiert und als Proxyserver zwischen Protractor und dem Browser agiert\cite{selenium}. Selenium unterstützt alle großen Webbrowser: aktuell die aktuellsten Versionen von Firefox, Internet Explorer ab Version 7, Safari ab Version 5.1, Opera und Chrome\cite{selenium-browsers}. Vom Einsatz von PhantomJS (s. Abschnitt \ref{sec:PhantomJS}) zusammen mit Protractor wird ausdrücklich abgeraten, da es hier Berichten zufolge häufig zu Abstürzen und abweichendem Verhalten kommt\cite{protractor-browser}. Laut eigener Aussage wird Selenium automatisch zusammen mit Protractor installiert und ist nach Aufruf von \texttt{webdriver-manager update} und \texttt{webdriver-manager start} ohne weitere Konfiguration lauffähig\cite{protractor-index}.

Protractor nutzt als Framework für die Testbeschreibung standardmäßig Jasmine (s. Abschnitt \ref{sec:Jasmine}), unterstützt out-of-the-box aber auch Mocha (s. Abschnitt \ref{sec:Mocha}). Die nachfolgenden Beispiele nutzen daher auch Jasmine. Das eingesetzte Testframework, die Adresse unter welcher der Selenium-Server angesprochen wird, Testdateien, Timeouts, für den Test zu verwendende Browser und weitere Feineinstellungen werden in einer Konfigurationsdatei (s. Listing \ref{lst:protractor-conf}) konfiguriert.

\begin{figure}[H]
	\lstinputlisting[caption={Beispiel einer Protractor-Konfiguration (adaptiert nach \cite{protractor-tutorial})}, label=lst:protractor-conf]{lst/protractor.conf.js}
\end{figure}

Üblicherweise hat jede zu testende Seite eine eigene Testsuite und jeder Testfall ist eine eigene Spec (s. Listing \ref{lst:protractor-spec}). Vor der eigentlichen Testdurchführung muss die jeweilige Seite aufgerufen werden: hierzu dient die durch Protractor bereitgestellte Funktion \texttt{browser.get(url)}. Es bietet sich an, diese in \texttt{beforeEach()} auszuführen, einer Funktion die durch Jasmine vor jedem Spec aufgerufen wird. Auf Elemente kann mit der Funktion \texttt{element} zugegriffen werden, welcher ein Locator übergeben wird. Locator sind ein durch Protractor definiertes Konstrukt und beschreiben, wie das Element gefunden werden kann. Um mit den gefundenen Elementen zu interagieren werden verschiedene Funktionen bereitgestellt: beispielsweise \texttt{sendKeys} zur Zeicheneingabe, \texttt{click} zum Simulieren eines Mausklicks oder \texttt{getText} um den Elementinhalt zu ermitteln.

\begin{figure}[H]
	\lstinputlisting[caption={Beispiel einer Spec für Protractor (aus \cite{protractor-tutorial})}, label=lst:protractor-spec]{lst/protractor.spec.js}
\end{figure}





\subsubsection{PhantomJS}
\label{sec:PhantomJS}
PhantomJS ist ein skriptbarer WebKit-Browser ohne Benutzeroberfläche und ist über eine JavaScript-API ansteuerbar. Er bietet native Unterstützung für Webstandards wie DOM, CSS-Selektoren, JSON, HTML-Canvas und SVG.\cite{phantomjs-index} PhantomJS steht nicht als npm-Paket zur Verfügung, sondern lässt sich lediglich als Binary installierten\cite{phantomjs-faq}. PhantomJS ist auch eine Laufzeitumgebung für JavaScript, so dass für ihn bestimmte Skripte nicht in Node.js ausgeführt werden können, sondern nur in PhantomJS\cite{phantomjs-quickstart}.

Für das Laden von Webseiten sind Page-Objekte zuständig, über welche eine URL geladen werden kann, Screenshots gespeichert werden können oder auf DOM-Eigenschaften zugegriffen werden kann. Ein Beispiel findet sich in Listing \ref{lst:phantomjs}. JavaScript-Code kann im Kontext einer geladenen Seite mit der \texttt{evaluate}-Funktion ausgeführt werden; die Ausführung erfolgt in einer Sandbox und kann somit nicht auf Objekte, Variablen oder Funktionen außerhalb des Kontexts zugreifen.\cite{phantomjs-quickstart}

\begin{figure}[H]
	\lstinputlisting[caption={Beispiel eines Seitenaufrufs mit PhantomJS (aus \cite{phantomjs-quickstart})}, label=lst:phantomjs]{lst/phantom.js}
\end{figure}

Ein großer Nutzungsbereich von PhantomJS liegt im Testen von Webanwendungen; geeignet ist es beispielsweise für den Einsatz in Kommandozeilenumgebungen oder Continuous-Integration-Systemen. PhantomJS an sich ist kein Testframework, sondern dient der Ausführung eines beliebigen Testframeworks. Es existieren Frameworks welche speziell auf PhantomJS aufbauen und komfortable Funktionalitäten für Testzwecke zur Verfügung stellen: z.\,B. CasperJS (s. Abschnitt \ref{sec:CasperJS}) oder WebSpecter, welches sich allerdings noch in einer frühen Entwicklungsphase befindet.\cite{phantomjs-testing}

\subsubsection{CasperJS}
\label{sec:CasperJS}
CasperJS ist ein Framework für Navigation und Test in PhantomJS. Es steht unter \texttt{casperjs} im npm-Repository zur Verfügung.\cite{casperjs-index} Trotz dessen ist es nicht unter Node.js lauffähig\cite{casperjs-faq}, sondern benötigt Python\cite{casperjs-installation}.

Es ermöglicht die Erstellung von komplexen Navigationsszenarien unter Benutzung von High-Level-Funktionen. Hierzu werden u.\,a. die Funktionen \texttt{casper.start}, \texttt{casper.then}, \texttt{casper.thenOpen} und \texttt{casper.back} sequentiell aufgerufen. Das Szenario kann dann mittels \texttt{casper.run} aufgerufen werden und wird nacheinander abgearbeitet. Es stehen diverse Funktionen wie \texttt{casper.click}, \texttt{casper.fill} oder \texttt{casper.evaluate} zur DOM-Manipulation zur Verfügung. Für alle Browser-Operationen und DOM-Manipulationen wird als Browser PhantomJS genutzt, welcher von CasperJS gestartet wird. Die Ansteuerung von PhantomJS wird durch CasperJS vereinfacht, wodurch sich einfacher wartbarerer Code ergibt\cite{casperjs-better-phantomjs}. Ein beispielhaftes Navigationsszenario findet sich in Listing \ref{lst:casper-scenario}.\cite{casperjs-index}

\begin{figure}[H]
	\lstinputlisting[caption={Beispiel eines Navigationsszenarios mit CasperJS (adaptiert nach \cite{casperjs-index})}, label=lst:casper-scenario]{lst/casper-scenario.js}
\end{figure}

CasperJS enthält auch ein einfaches Testframework. Ein Test wird durch Aufruf der Funktion \texttt{casper.test.begin} definiert und mit \texttt{test.done} beendet. Er gilt als erfolgreich, wenn er ohne fehlgeschlagene Assertions beendet wurde. Navigationsszenarien und Tests können verschachtelt werden, so dass mit CasperJS auch End-To-End-Tests durchgeführt werden können.\cite{casperjs-index, casperjs-test} Ein einfacher Test ist in Listing \ref{lst:casper-test} zu finden.

\begin{figure}[H]
	\lstinputlisting[caption={Beispiel eines Tests mit CasperJS (aus \cite{casperjs-test})}, label=lst:casper-test]{lst/casper-test.js}
\end{figure}


\subsubsection{Sinon}
\label{sec:Sinon}
Sinon ist ein Framework für die Realisierung von Fakes, also Mocks, Stubs und Spies, in JavaScript und arbeitet mit jedem Unit-Test-Framework zusammen. Es ist als \texttt{sinon} über npm verfügbar.\cite{sinonjs-index}

\paragraph*{Spies}
Spies sind Funktionen die relevante Aufrufdaten aufzeichnen: Argumente, Rückgabewert, geworfene Exceptions und den Aufrufer. Ein Spy kann sowohl als anonyme Funktion - durch Aufruf von \texttt{sinon.spy()} - als auch als Wrapper für existierende Methoden - dann durch Aufruf von \texttt{sinon.spy(object, 'method')} für das Überschreiben von \texttt{object.method()}.\cite{sinonjs-spies} Ein Beispiel für einen Spy auf einer anonymen Funktion findet sich in Listing \ref{lst:sinon-spy}.

\begin{figure}[H]
	\lstinputlisting[caption={Beispiel eines anonymen Spy in Sinon (aus \cite{sinonjs-index})}, label=lst:sinon-spy]{lst/sinon-spy.js}
\end{figure}

\paragraph*{Stub}
Stubs sind Spies mit einem vorprogrammierten Verhalten. Hierzu verfügen sie über die komplette Spy-API und zusätzliche Methoden, mit welchen ihr Verhalten angepasst werden kann. Anders als bei Spies wird bei einem Stub, welcher eine existierende Funktion überschreibt, diese nicht ausgeführt. Sie können benutzt werden um ein bestimmtes Verhalten von Funktionen zu provozieren, z.\,B. durch das Werfen von Fehlern, oder um zu verhindern, dass Funktionen ausgeführt werden, z.\,B. \texttt{XMLHttpRequest} damit kein HTTP-Request abgesetzt wird. Ein Aufruf von \texttt{sinon.stub().throws()} erzeugt beispielsweise einen anonymen Stub, welcher bei Aufruf eine Exception wirft.\cite{sinonjs-stubs}

\paragraph*{Mocks}
Mocks sind Stubs, welche zusätzlich vorprogrammierte Erwartungen haben. Ein Mock ist somit ein Stub, welcher Assertions enthält. Es wird empfohlen, dass maximal ein Mock pro Unittest existiert. Ein Beispiel für einen Mock, in welchem erwartet wird, dass die gemockte Methode einmal aufgerufen wird und diese eine Exception werfen soll, findet sich in Listing \ref{lst:sinon-mock}. Das Eintreffen der definierten Erwartungen wird letztlich durch Aufruf von \texttt{verify()} überprüft - ein Nicht-Zutreffen führt wie bei Assertions zum Fehlschlag des Tests.\cite{sinonjs-mocks}

\begin{figure}[H]
	\lstinputlisting[caption={Beispiel eines Mocks in Sinon (aus \cite{sinonjs-mocks})}, label=lst:sinon-mock]{lst/sinon-mock.js}
\end{figure}

Sinon ermöglicht es, mit der Funktion \texttt{sinon.restore} alle Fakes, die auf einem übergebenen Objekt definiert wurden, zurückzusetzen. Es bietet außerdem die Möglichkeit \textit{Sandboxes} anzulegen, in welchen alle angelegten Fakes abgelegt werden. Dies erleichtert das Aufräumen und Entfernen aller Fakes, da dies nun nicht mehr einzeln geschehen muss, sondern ein Aufruf von \texttt{sandbox.restore} genügt.\cite{sinonjs-sandboxes}

\subsubsection{ngMock}
\label{sec:ngMock}
Bei ngMock handelt es sich um ein AngularJS-Modul, mit welchem andere Komponenten in Unit-Tests injiziert und gemockt werden können. Außerdem erweitert es diverse AngularJS-Kernservices, so dass diese in Testcode kontrolliert und inspiziert werden können. Es steht im npm-Repository unter \texttt{angular-mocks} zur Verfügung und muss in der Konfiguration des verwendeten Test-Runners so eingebaut werden, dass es nach \texttt{angular.js} geladen wird.\cite{angular-ngMock}

Die Funktionsweise basiert auf den Methoden \texttt{angular.mock.module}, welche die übergebenen Module lädt\cite{angular-ngMockModule}, und \texttt{angular.mock.inject}, welche eine übergebene Funktion in eine Injizierbare wrapt, eine neue Injector-Instanz erstellt und die angegebenen Abhängigkeiten injiziert\cite{angular-ngMockInject}. Methoden von injizierten Abhängigkeiten können, beispielsweise mit Sinon (s. Abschnitt \ref{sec:Sinon}), mit Spies oder Mocks ersetzt werden.

\begin{figure}[H]
	\lstinputlisting[caption={Beispiel eines Tests mit injizierten Abhängigkeiten mit ngMock (adaptiert nach \cite{watmore})}, label=lst:ngMock]{lst/ngMock.js}
\end{figure}

Im Beispiel (s. Listing \ref{lst:ngMock}) wird das Module \texttt{app} geladen und die Services \texttt{SimpleService} und \texttt{\$log} in die Testfunktion injiziert. Dadurch kann ein Spy auf \texttt{\$log.info} gesetzt werden und hierdurch das korrekte Logging von \texttt{SimpleService.DoSomething} validiert werden.





\newpage
\subsection{Auswahlentscheidung}
\label{sec:auswahlentscheidung}
Für die Entscheidungsfindung wird eine Entscheidungsmatrix (s. Tabelle \\ref{tbl:entscheidungsmatrix}) erstellt, in welcher die gefundene Software mit Blick auf die Anforderungen bewertet wird. Ein rotes Kreuz bedeutet, dass die Software der Anforderung widerspricht und die Erfüllung der Anforderung unmöglich macht. Ein grüner Haken steht für ein Erfüllen der Anforderung. Eine leere Zelle bedeutet, dass die Software die Anforderung weder explizit erfüllt noch ihr widerspricht; es besteht keine Relevanz. Die ausgewählte Softwaremenge erfüllt eine Anforderung somit nur dann, wenn lediglich grüne Haken oder leere Zellen vorhanden sind, also keine Einzelsoftware der Anforderung widerspricht.

Wie in Abschnitt \ref{sec:anforderungsanalyse} bereits beschrieben werden zur Erfüllung von Anforderung A1 ein Testrunner, ein Testframework sowie eine Assertion-Bibliothek benötigt; zur Erfüllung von Anforderung A2 zusätzlich ein Tool zur Browsersteuerung. Hierzu werden die gefundenen Programme kategorisiert. Die endgültig ausgewählte Softwaremenge muss mindestens ein Exemplar jeder Kategorie enthalten, damit die Anforderungen A1 und A2 erfüllt werden.

%\begin{table}[H] \centering
	%\begin{tabular}{@{} cl*{11}c @{}}
	%	& & \rot{Karma} & \rot{Mocha} & \rot{AVA} & \rot{QUnit} & \rot{Intern} & \rot{Jasmine} & \rot{Chai} & \rot{Protractor} & \rot{PhantomJS} & \rot{CasperJS} & \rot{Sinon} & \rot{ngMock} \\\hline
	%	& \textit{1a} - Komponententest & \OK & \OK & \OK & \OK & \OK & \OK & & & & & &  \\
	%	
	%\end{tabular}
	\begin{tabularx}{\textwidth}{@{}r@{\hskip 6pt}X|ccccccccccccc}
		&& \multicolumn{12}{c}{Software} \\ &&  \rot{\footnotesize{Karma}} & \rot{\footnotesize{Mocha}} & \rot{\footnotesize{AVA}} & \rot{\footnotesize{QUnit}} & \rot{\footnotesize{Intern}} & \rot{\footnotesize{Jasmine}} & \rot{\footnotesize{Chai}} & \rot{\footnotesize{Protractor}} & \rot{\footnotesize{PhantomJS}} & \rot{\footnotesize{CasperJS}} & \rot{\footnotesize{Sinon}} & \rot{\footnotesize{ngMock}} & \rot{\footnotesize{Istanbul}} \\ \hline
		%									Ka		Mo		Av		QU		In		Ja		Ch		Pr		Ph		Ca		Si		ng		Is
	    \multirow{5}{*}{\centering\rot{\footnotesize{Kategorie}}}
		&	\footnotesize{Testrunner}	& \ok	& \ok	& \ok	& \ok	& \ok	& \ok	& 		& 		&		& \ok	&		&  		&\\
	    &	\footnotesize{Testframework}
							    		&		& \ok	& \ok	& \ok	& \ok	& \ok	&		&		&		& \ok	&		& 		&\\
	    &	\footnotesize{End-To-End}	
									    & 		&		&		&		& \ok	&		&		& \ok	& \ok	& \ok	&		&  		&\\
	    &	\footnotesize{Assertion}	& 		&		& \ok	& \ok	& \ok	& \ok	& \ok	& 		& 		& \ok	& 		&  		&\\
	    &	\footnotesize{Sonstige}		&		&		&		&		&		&		&		&		&		&		& \ok	& \ok	& \ok \\\hline\hline
	    
	    \footnotesize{A3}
	     &	\footnotesize{AngularJS-Support}	
									    & \ok	& \ok	& \nok\footnote{\url{https://github.com/angular/angular.js/issues/13971}}
																& \ok	& \ok	& \ok	& \ok	& \ok	& \ok	& \ok
																													    & \ok	& \ok	& \ok \\
		\footnotesize{A4}
		 & \footnotesize{Testausführung}& \ok	& 		& 	  	& 		& 		& \ok	& 		&		&		& \nok	&		&		& \\
		
		\footnotesize{A5}
		 & \footnotesize{Open Source \& kommerziell}
										& \ok	& \ok	& \ok	& \ok	& \ok	& \ok	& \ok	& \ok	& \ok	& \ok	& \ok	& \ok	& \ok \\
		\footnotesize{A6}
		 & \footnotesize{Wartung\footnote{Zur Auswertung dieser Anforderung wird das jeweilige Github-Repository herangezogen.}}
										& \ok	& \ok	& \ok	& \ok	& \ok	& \ok	& \ok	& \ok	& \nok\footnote{1.841 offene Issues; 2 Issues geschlossen in der letzten Woche \cite{phantomjs-issues,phantomjs-pulse}}
																												& \ok	& \ok	& \ok	& \ok \\ 
		%11 & \textbf{ECMA-262 5.1}		& 		& \ok	&		&		&		&		& \ok	&		&		&		&		& \\

		\footnotesize{A7}
			 & \footnotesize{BDD-Stil}	&		& \ok	& \nok	& \nok	& \ok	& \ok	& \ok	&		&		& \nok	&		& 		&\\
		\footnotesize{A8} 
		& \footnotesize{Einarbeitung \& Verwendung}
										& \ok	& \ok	& \ok	& \ok	& \nok\footnote{Warten auf asynchrone AngularJS-Komponenten muss bei End-To-End-Tests manuell implementiert werden.}
																				& \ok	& \ok	& \ok	& \ok	& \nok\footnote{Warten auf asynchrone AngularJS-Komponenten muss manuell implementiert werden.}
																														& \ok	& \ok	& \ok \\
		\footnotesize{A9}
		 & \footnotesize{Flexibilität}& 		& \ok	& \nok	& \nok	& \ok	& \nok	& 		&		&		& \nok	&		& 		& \\
		\footnotesize{A10}
		 & \footnotesize{Ausgabe-Konfiguration}
										& \ok	& \ok	& \ok	& \ok	& \ok	& \ok	&		&		& 		& \nok	&		& 		& \ok \\
		\footnotesize{A11}
		 & \footnotesize{Screenshots}	&		&		&		&		& \nok	&		&		& \ok	& \ok	& \ok	&		& 		&\\
		\footnotesize{A12}
		 & \footnotesize{Spies, Stubs, Mocks}
										&		&		&		&		&		&		&		&		&		& 		& \ok	& \ok	& \\
		\footnotesize{CB1}
		&	\footnotesize{Windows 7}			& \ok	& \ok	& \ok	& \ok	& \ok	& \ok	& \ok	& \ok	& \ok	& \ok	& \ok	& \ok 	& \ok \\
		\footnotesize{CB2}
		& \footnotesize{benötigte Software}	& \ok	& \ok	& \ok	& \ok	& \ok	& \ok	& \ok 	& \ok\footnote{Zur Ausführung wird zwar Selenium benötigt, dies wird aber automatisiert mitinstalliert. Die Software erfüllt die Anforderung daher.}
		& \ok	& \nok	& \ok	& \ok 	& \ok \\
		\footnotesize{CB3}
		& \footnotesize{Node.js 9.6.2}& \ok	& \ok	& \ok 	& \ok	& \ok	& \ok	& \ok	& \ok	&  \ok	& \ok	& \ok	& \ok	& \ok \\
		
		\footnotesize{CB4}
		& \footnotesize{npm}			& \ok	& \ok	& \ok	& \ok	& \ok	& \ok	& \ok	& \ok	& \nok	& \ok	& \ok	& \ok	& \ok \\		
		\footnotesize{CB5}
		& \footnotesize{Forcierte Beschreibung}
		&		& \ok	& \nok	& \ok	& \ok	& \ok	& 		& 		&		& \ok	& 		&		& \\		
				\footnotesize{CB6}
				& \footnotesize{Code-Coverage}		
				& 		& 		& 		& 		& \ok	&		&		&		&		&		&		& 		& \ok\\ 
				\footnotesize{CB7}
				& \footnotesize{keine proprietären Dateien}
				& \ok	& \ok	& \ok	& \ok	& \ok	& \ok	& \ok	& \ok	& \ok	& \ok	& \ok	& \ok	& \ok	 
					
	\end{tabularx}
	\captionof{table}{Entscheidungsmatrix zur Softwareauswahl\label{tbl:entscheidungsmatrix}}
	
	Die Auswertung der Entscheidungsmatrix zeigt, dass AVA, QUnit, Intern, Jasmine, PhantomJS sowie CasperJS nicht eingesetzt werden können, da sie Anforderungen widersprechen. Die Verbleibenden bilden die Basis für die zu treffende Auswahl aufgrund der Anforderungen. Es zeigt sich, dass nur durch den Einsatz sämtlicher verbleibender Software sämtliche Anforderungen abgedeckt werden können. Es wird somit folgende Auswahl getroffen:
	
	Als Test-Runner wird Karma eingesetzt, welches die benötigten JavaScript-Dateien importiert, bereitstellt und aus der Konsole heraus die in Mocha implementierten Tests ausführt. Die sich aus der Matrix ergebende Kombination von Mocha mit der Assertionbibliothek Chai wird von \textcite{mocha-index} bereits in der Softwaredokumentation aufgeführt, wodurch hier von einem problemlosen Zusammenarbeiten auszugehen ist. Für die Ermittlung der Code-Coverage wird Istanbul eingesetzt; die Kombination des Einsatzes mit Mocha wird in  \cite{istanbul} explizit empfohlen, so dass auch hier von keinerlei Problemen auszugehen ist.
	
	Für die Realisierung von End-To-End-Tests wird auf das vom AngularJS-Team entwickelte Protractor zurückgegriffen und somit auch der Empfehlung aus der offiziellen AngularJS-Dokumentation  \cite{angular-e2e} gefolgt. Die abweichende Kombination von Protractor mit Mocha und Chai, statt Jasmine, wird von vielen Entwicklern genutzt und unter Anderem in  \cite{protractormocha-1} und  \cite{protractormocha-2} beschrieben, so dass auch hier von keinen Problemen auszugehen ist.
	
	Für die Realisierung von Spies, Stubs und Mocks in Komponententests wird auf Sinon und ngMock gesetzt.
%\end{table}



\newpage
\section{Evaluierung}
Im folgenden Kapitel wird die getroffene Softwareauswahl evaluiert, indem für ein gegebenes, bereits existierendes AngularJS-Projekt im Commerzbank-Umfeld prototypisch Tests implementiert werden. Hierbei werden nicht Tests für alle Komponenten und Funktionalitäten der Anwendung entwickelt, sondern nur so viele, bis sich die getroffene Softwareauswahl als problemlos einzusetzen herausstellt.

\subsection{Projektbeschreibung}
Das Projekt GFB stellt ein digitales Formular bereit, mittels welchem Gefährdungsbeurteilungen erfasst werden können. Dies sind Dokumente, welche beispielsweise von Fachkräften für Arbeitssicherheit nach Betriebsbegehungen erstellt werden und in denen festgestellte Mängel aufgeführt werden.\cite{gfb}

Die Anwendung ist eine Single-Page-Application und wurde unter Nutzung von AngularJS entwickelt. Bereitgestellt wird sie in einer Microsoft SharePoint-Umgebung, wodurch sich einige Besonderheiten ergeben: z.\,B. die Nutzung von SOAP zur Abfrage von WebServices. Hierdurch funktioniert die Anwendung auch ausschließlich im Internet Explorer und nicht im zweiten Commerzbank-Standardbrowser Firefox. Für die Anwendung existieren keine automatisierten Tests; jegliches Testen wird auf der Oberfläche im Testsystem durchgeführt.\cite{gfb}

\subsection{Implementierung}
Die Implementierung der Tests erfolgt in drei Iterationen: Zunächst wird eine Testumgebung, bestehend aus Karma, Mocha, Chai, Sinon und Istanbul, aufgesetzt und gezeigt, dass diese funktioniert. Anschließend werden prototypische Komponententests für alle verschiedenen Komponententypen (Controller, Service, Filter, Direktive) in AngularJS implementiert. Abschließend werden Systemtests als End-To-End-Tests umgesetzt; auch dies geschieht prototypisch und hat keine hundertprozentige Abdeckung zum Ziel.

\subsection{Auswertung}

\newpage
\section{Ausblick}
\subsection{Einsatz von automatisierten Tests im Commerzbank-Konzern}


\subsection{Mögliche Verbesserungen durch Einsatz von Angular statt AngularJS}
Angular bietet im Vergleich zum älteren AngularJS einige Verbesserungen im Hinblick auf automatisierte Testbarkeit, welche in diesem Kapitel oberflächlich betrachtet werden sollen.

Angular wird zusammen mit der Angular CLI ausgeliefert, einem Kommandozeilentool zur Vereinfachung verschiedener häufig benötigter Abläufe. Der integrierte Assistent zur Generierung der Grundstruktur neuer Angular-Projekte bindet in diese auch eine Unterstützung für automatisierte Tests ein: ein konfiguriertes Karma sowie passende Ordnerstrukturen. Dieses kann umkonfiguriert werden, so dass auch die Nutzung von Mocha und Chai, statt dem standardmäßigen Jasmine, möglich ist.\cite{ng2-cli}

Es bietet die \textit{Angular Testing Utilities}, wobei es sich um Klassen und Funktionen handelt, welche die Testdurchführung erleichern. Sie ermöglichen es, ein \texttt{TestBed} zu erstellen, mit welchem ein Testmodul geschaffen wird, welches die zu testende Komponente enthält. Aus diesem Testmodul kann ein \texttt{TestFixture} kreiert werden, welches Zugriff auf die zu testene Komponente sowie auf das, der Komponente zugeordnete, DOM-Element bietet. Dies bietet im Vergleich zu Tests mit ngMock in AngularJS den Vorteil einer besseren Lesbarkeit und dem einfacheren Zugriff auf interne Funktionen und Eigenschaften. Ein beispielhafter Test unter Nutzung der Angular Testing Utilities findet sich in Listing \ref{lst:ng2-test}. \cite{ng2-test}

Zusammenfassend lässt sich sagen, dass die Realisierung von automatisierten Tests in neueren Versionen von Angular mit weniger Problemen und deutlich einfacher möglich ist, so dass sich mögliche Anfangshindernisse und -schwellen verringern und so den Einstieg eines Entwicklers in automatisierte Tests erleichern. 

\begin{figure}[H]
	\lstinputlisting[caption={Beispielhafter Komponententest einer \textit{Component} in Angular; geschrieben in TypeScript (aus \cite{ng2-test})}, label=lst:ng2-test]{lst/ng2-test.js}
\end{figure}

\subsection{Fazit}



\onecolumn
% einfacher Zeilenabstand
\singlespacing
% Literaturliste soll im Inhaltsverzeichnis auftauchen
\newpage
\addcontentsline{toc}{section}{Literaturverzeichnis}
% Literaturverzeichnis anzeigen
\renewcommand\refname{Literaturverzeichnis}
\printbibliography

%% Index soll Stichwortverzeichnis heissen
% \newpage
% % Stichwortverzeichnis soll im Inhaltsverzeichnis auftauchen
% \addcontentsline{toc}{section}{Stichwortverzeichnis}
% \renewcommand{\indexname}{Stichwortverzeichnis}
% % Stichwortverzeichnis endgueltig anzeigen
% \printindex

\nomenclature{API}{Application Programming Interface}
\newpage
%\addcontentsline{toc}{section}{Stichwortverzeichnis}
%\printnomenclature
\fancyhead[L]{Glossar}
\section*{Glossar}
\textbf{Abkürzungen} \\
\begin{tabularx}{\linewidth}{@{}>{\bfseries}l@{\hspace{.5em}}X@{}}
	BaFin		&	Bundesanstalt für Finanzdienstleistungsaufsicht \\
	DOM			&	Document Object Model \\
	HTTP		&	Hypertext Transfer Protocol \\
	JSON		&	JavaScript Object Notation \\
	MVVM		&	Model-View-ViewModel, ein Entwurfsmuster \\
	MVC			&	Model-View-Controller, ein Entwurfsmuster \\
	REST		&	Representational State Transfer \\
	SEF			&	Software Engineering Framework (der Commerzbank) \\
\end{tabularx}

\textbf{Begriffe} \\
\begin{tabularx}{\linewidth}{@{}>{\bfseries}l@{\hspace{.5em}}X@{}}
	falsy			&	Ein in JavaScript als false angesehener Wert: false, 0, \glqq\grqq (leerer String), null, undefined und NaN. \\
	jQuery			&	jQuery ist eine JavaScript-Bibliothek für DOM-Manipulation, Event-Handling und Animation. \\
	truthy			&	Ein in JavaScript als true angesehener Wert: alle nicht falsy Werte, auch \glqq 0\grqq, \glqq false\grqq{} oder leere Funktionen, Arrays, Objekte. \\
	WebSocket		&	WebSocket ist eine bidirektionale Erweiterung von HTTP. \\
	Whitebox-Test	&	Ein Whitebox-Test ist ein Test, welcher unter Kenntniss der inneren Funktionsweise entwickelt wird. \\
	
\end{tabularx}

%\begin{description}
%	\item[falsy] Ein in JavaScript als \texttt{false} angesehener Wert: \texttt{false}, \texttt{0}, \texttt{\glqq\grqq }(leerer String), \texttt{null}, \texttt{undefined} und \texttt{NaN}.
%	\item[jQuery] jQuery ist eine JavaScript-Bibliothek für \textit{DOM}-Manipulation, Event-Handling und Animation.
%	\item[truthy] Ein in JavaScript als \texttt{true} angesehener Wert: alle nicht \textit{falsy} Werte, auch \texttt{\glqq 0\grqq}, \texttt{\glqq false\grqq{}} oder leere Funktionen, Arrays, Objekte.
%	\item[WebSocket] WebSocket ist eine bidirektionale Erweiterung von \textit{HTTP}.
%	\item[Whitebox-Test] Ein Whitebox-Test ist ein Test, welcher unter Kenntnis der inneren Funktionsweise entwickelt wird.
%\end{description}

\onehalfspacing
% evtl. Anhang
\newpage
\addcontentsline{toc}{section}{Anhang}
\section*{Anhang}
\fancyhead[L]{Anhang} %Kopfzeile links
\renewcommand{\thesubsection}{\Alph{subsection}}
\setcounter{subsection}{0}
\subsection*{Begriffsklärung}
\newpage
\subsection*{Anhang}\label{anhang}

% Eidesstattliche Erklärung
\addcontentsline{toc}{section}{Eidesstattliche Erklärung}
\include{erklaerung}

% leere Abschlussseite
\newpage
\thispagestyle{empty} % erzeugt Seite ohne Kopf- / Fusszeile
\section*{ }

\end{document}