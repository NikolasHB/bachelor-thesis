\thispagestyle{empty}

%\begin{figure}[t]
% \includegraphics[width=0.6\textwidth]{abb/fh_koeln_logo}
%\end{figure}

\begin{figure}[t]
 \centering
 \includegraphics[width=0.8\textwidth]{abb/logo}
\end{figure}


\begin{verbatim}


\end{verbatim}

\begin{center}
\Large{Hochschule Bremen}\\
\Large{ZIMT}\\
\end{center}


\begin{center}
\Large{Fakultät für Elektrotechnik und Informatik}
\end{center}
\begin{verbatim}




\end{verbatim}
\begin{center}
\doublespacing
\textbf{\LARGE{\titleDocument}}\\
\singlespacing
\begin{verbatim}

\end{verbatim}
\textbf{{~\subjectDocument~-~Schwerpunkt Technische Informatik}}
\end{center}
\begin{verbatim}

\end{verbatim}
\begin{center}

\end{center}
\begin{verbatim}

\end{verbatim}
\begin{center}
\textbf{zur Erlangung des akademischen Grades \\ Bachelor of Science}
\end{center}
\begin{verbatim}






\end{verbatim}
\begin{flushleft}
\begin{tabular}{llll}
\textbf{Thema:} & & Penetrationstest eines sicherheitskritischen Software-Systems  & \\
& & im Finanzsektor zur Identifizierung von Schwachstellen & \\
&& unter Einsatz des Metasploit-Frameworks & \\
& & \\
\textbf{Autor:} & & Martin Suckau <martin.suckau@web.de> & \\
& & MatNr. 353129 & \\
& & \\
\textbf{Version vom:} & & \today &\\
& & \\
\textbf{1. Betreuer:} & & Prof. Dr. Richard Sethmann &\\
\textbf{2. Betreuer:} & & Prof. Dr. Uwe Meyer &\\
\end{tabular}
\end{flushleft}