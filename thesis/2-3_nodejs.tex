\subsubsection{Laufzeitumgebung}
Node.js ist eine Laufzeitumgebung, mit der JavaScript serverseitig, also ohne Webbrowser, ausgeführt werden kann\cite[][1]{hughes2012einfuehrungnodejs}. Somit ist es möglich, JavaScript nicht nur für die Darstellung von Benutzeroberflächen im Webbrowser zu Nutzen, sondern auch als Backend-Sprache oder zur Unterstützung von Entwicklungsprozessen auf Continuous-Integration-Servern oder Entwicklerarbeitsplätzen.

Intern nutzt Node.js die JavaScript-Engine \textit{Chrome V8} \cite{nodejs}, welche von Google als Open-Source-Software veröffentlicht wurde. V8 kommt auch im weitverbreiteten Webbrowser \textit{Google Chrome} zum Einsatz und implementiert den JavaScript-Standard ECMAScript wie in ECMA-262 spezifiziert\cite{chromev8}. ECMA-262 ist der im Juni 2016 veröffentlichte und zurzeit aktuellste JavaScript-Standard\cite{ecma262}. Somit bietet Node.js alle spezifizierten und von Google Chrome unterstützten Sprachfunktionalitäten. Es eignet sich daher auch für den Test von für Webbrowser entwickelte Webanwendungen.

\subsubsection{Node Package Manager (npm)}
Der Node Package Manager (npm) ist der zusammen mit Node.js installierte Paketmanager sowie Buildmanager für JavaScript. Unter npm wird außerdem die \textit{npm Registry}, also die zentrale Ablage von mit npm verwendeten JavaScript-Paketen, verstanden, auf welche der Node Package Manager zugreift. \cite{npm-about} Die npm Registry enthält über 180.000 Pakete und ist damit das größte öffentliche Softwarerepository\cite{modulecount}.

Grundlegend funktioniert die Paketverwaltung mit einer JSON-Konfigurationsdatei, der \texttt{package.json} (vgl. Listing \ref{lst:package-json}). Die Datei enthält den Namen sowie die Version des Pakets, für welches sie angelegt wurde, sowie optional weitere Metadaten wie Beschreibung, Autor und Referenzlinks auf Bugtracker. Außerdem werden hier Abhängigkeiten angegeben, die zur Ausführung (\texttt{dependencies}) oder zur Entwicklung (\texttt{devDependencies}) in diesem Paket benötigt werden.\cite{npm-packagejson} Die angegebenen Abhängigkeiten werden von npm automatisiert heruntergeladen und im Ordner \texttt{node\_modules} abgelegt, von wo aus sie in die JavaScript-Anwendung eingebunden werden können. Auch transitive Abhängigkeiten werden von npm aufgelöst.\cite{npm-install}

\begin{figure}[H]
	\lstinputlisting[caption=Beispiel einer package.json, label=lst:package-json]{lst/package.json}
\end{figure}

Neben der Paketverwaltung kann npm auch zum Build als Taskrunner eingesetzt werden. Hiermit kann der Buildprozess eines Paketes automatisiert werden, z.\,B. durch die automatisierte Ausführung von Tests oder dem Aufrufen von Compilern. Hierzu werden in der \texttt{package.json} Skripte angegeben. Diese bestehen aus einem Skript-Namen und dem auszuführenden Befehl. Im Beispiel (siehe Listing \ref{lst:packageWithScripts-json}) werden drei Skripte definiert:\cite{cirkel-npmAsABuildTool}
\begin{itemize}
	\item \glqq lint\grqq~ führt das Kommando \texttt{jshint **.js} aus. Dies dient dem Überprüfen von JavaScript-Dateien auf statische Programmierfehler\cite{jshint-about}.
	\item \glqq build\grqq~ führt das Kommando \texttt{browserify [...]} aus. Dieses dient dem Zusammenfügen von mittels \texttt{require} eingebundenen JavaScript-Dateien in eine konkatenierte Datei\cite{browserify-about}.
	\item \glqq test\grqq~ führt das Kommando \texttt{mocha [...]} aus. Mocha ist ein Test-Runner (siehe auch Abschnitt \ref{sec:Mocha}).
\end{itemize}

Angegebene Skripte können auch automatisch in sogenannten Hooks (\textit{Pre} und \textit{Post Hooks}) ausgeführt werden. Im Beispiel (siehe Listing \ref{lst:packageWithScripts-json}) sind folgende Hooks definiert:\cite{cirkel-npmAsABuildTool}
\begin{itemize}
	\item \glqq prepublish\grqq{} wird vor der Ausführung von \texttt{publish}, welches ein npm Standard-Skript ist und das Paket in der npm Registry veröffentlicht\cite{npm-publish} und das benutzerdefinierte Skript \texttt{build} ausführt. Durch die Ausführung von \texttt{build} werden automatisch auch \texttt{prebuild} und \texttt{postbuild} ausgeführt.
	\item \glqq prebuild\grqq{} wird vor Ausführung des \texttt{build}-Skripts das Paket durch Ausführung von \texttt{test} überprüfen.
	\item \glqq pretest\grqq{} wird vor Ausführung von \texttt{test} mittels \texttt{lint} das Paket auf statische Fehler untersuchen.
\end{itemize}

\begin{figure}[H]
	\lstinputlisting[caption={Beispiel von Skripten in einer package.json (aus \cite{cirkel-npmAsABuildTool})}, label=lst:packageWithScripts-json]{lst/packageWithScripts.json}
\end{figure}

Die wohl populärste Funktion von Taskrunnern ist das automatisierte Beobachten des Dateisystems auf Änderungen. Häufig ist es wünschenswert, dass bei einer Dateiänderung automatisch ein entsprechender Buildprozess oder die Tests ausgeführt werden. Diese Funktionalität bietet npm im Gegensatz zu anderen Taskrunnern wie \textit{Gulp} oder \textit{Grunt} nicht nativ, sondern nur mithilfe eines Zusatzpakets. Ein entsprechendes Beispiel, welches bei Veränderung einer Datei im Paket-Ordner die JavaScript-Module zu einer Datei konkateniert\cite{browserify-about} und den JavaScript-Code des Paketes überprüft, findet sich in Listing \ref{lst:packageWithWatch-json}.\cite{cirkel-npmAsABuildTool}

\begin{figure}[H]
	\lstinputlisting[caption={Beispiel der Watch-Funktionalität in einer package.json (aus \cite{cirkel-npmAsABuildTool}, angepasst durch den Autor)}, label=lst:packageWithWatch-json]{lst/packageWithWatch.json}
\end{figure}