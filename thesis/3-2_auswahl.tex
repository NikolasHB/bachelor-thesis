\subsection{Anforderungsanalyse}
\label{sec:anforderungsanalyse}
An die auszuwählenden Frameworks oder Tools beziehungsweise eine Kombination Mehrerer stellen sich die folgenden Anforderungen:
\begin{enumerate}[label=\textbf{A\arabic*}]
	\item Es muss die Durchführung von Komponententests möglich sein.
	\item Es muss die Durchführung von End-To-End-Tests möglich sein.
	\item Es muss der Test von AngularJS-Anwendungen möglich sein.
	\item Es muss die Testausführung von Komponententests im Browser aus der Konsole heraus möglich sein.
	\item Sie muss Open-Source und für den kommerziellen Einsatz freigegeben sein.
	\item Sie soll durch eine Open-Source-Community aktiv gewartet werden; Fehler in ihr sollen behoben werden.
	%\item Es muss mindestens der aktuelle Commerzbank-Standard, ECMA-262 5.1, unterstützt werden\cite[][10]{coba-programmierrichtlinienJavaScript}.
	\item Der Testcode soll leicht lesbar sein\cite[][7]{coba-programmierrichtlinienAllgemein}; hierzu bietet sich der BDD-Stil an, den das Testframework somit unterstützen soll.
	\item Die Software soll eine geringe Einarbeitungszeit erfordern und problemlos zu verwenden sein. 
	\item Sie soll dem Entwickler Flexibilität beim Schreibe der Testfallimplementationen und Assertions bieten, damit dieser möglichst nah an seinen persönlichen Präferenzen arbeiten kann.
	\item Die Ausgabe der Testergebnisse muss konfigurierbar sein, um sie gegebenenfalls in Prozessen oder anderen Tools weiterverwenden zu können.
	\item Bei der Durchführung von End-To-End-Tests soll es möglich sein, die Ergebnisse mit Screenshots zu dokumentieren.
	\item Der Einsatz von Spies, Stubs und Mocks muss bei Unit-Tests möglich sein.
\end{enumerate}

\newpage
Aufgrund diverser regulatorischer Vorgaben, Prozesse und vorhandener technischer Systeme innerhalb des Commerzbank-Konzerns ergeben sich weitere Anforderungen:
\begin{enumerate}[label=\textbf{CB\arabic*}]
	\item Die Software muss unter Windows 7 funktionieren.
	\item Es darf keine zusätzliche Software zur Ausführung erforderlich sein; lediglich Node.js und Java sind zulässig, da diese über die Standardsoftwareverteilung der Commerzbank verfügbar sind.
	\item Wenn die Software Node.js benötigt, muss sie mit Version 6.9.2 lauffähig sein.
	\item Die Software muss über den Node Package Manager installierbar sein.
	\item Gemäß dem SEF der Commerzbank müssen Testfälle eindeutig beschrieben werden; diese Beschreibung auch kann durch Implementierung der Tests erfolgen\cite{coba-sef}. Das Testframework soll daher eine Beschreibung der Tests forcieren.
	\item Gemäß dem SEF sollen Komponententests \glqq einen möglichst großen Teil der Funktionalität abdecken\grqq{}\cite{coba-sef}. Zur Überprüfung dessen muss eine Möglichkeit zur Ermittlung der Code Coverage bestehen.
	\item Das Bearbeiten aller Dateien in der TFS-Quellcodeverwaltung soll möglich sein, es dürfen somit keine proprietären Binärdateien zum Einsatz kommen.
\end{enumerate}

Zur Erfüllung der Anforderung A1 werden ein \textit{Testrunner}, ein \textit{Testframework} zum Anlegen und Beschreiben der Testfälle sowie eine \textit{Assertion-Bibliothek} benötigt. Zur Erfüllung von Anforderung A2 wird zusätzlich ein \textit{Tool zur Browsersteuerung} benötigt.

\newpage
\subsection{Softwarerecherche}
Einen Überblick über zur Verfügung stehende Tools bieten \cite{unittest-overview} und Weitere.

\subsubsection{Karma}
\label{sec:Karma}
Karma ist ein Test-Runner für die Ausführung von JavaScript-Tests. Er wurde vom AngularJS-Team ins Leben gerufen und wird auf GitHub von einer Open-Source-Gemeinschaft weiterentwickelt.\cite{karma-index} Karma liegt als Paket \texttt{karma} im npm-Repository\cite{karma-faq}.

Karma ermöglicht die Nutzung diverser Testframeworks, wie Jasmine (s. Abschnitt \ref{sec:Jasmine}), Mocha (s. Abschnitt \ref{sec:Mocha}) oder QUnit (s. Abschnitt \ref{sec:QUnit}). Auch Continuous Integration Server wie Jenkins oder Travis werden unterstützt.\cite{karma-faq}

Grundlegend basiert Karma auf einem Client-Server-Prinzip, wobei Karma einen Webserver startet, welcher alle verbundenen Browser fernsteuert und in diesen die Tests ausführt. Ein Browser kann hierbei entweder manuell, also durch Aufruf der vom Karma-Server bereitgestellten URL, oder automatisiert, also indem der Browser durch Karma gestartet wird (vgl. Listing \ref{lst:karma-conf-js}), verbunden werden. In jeder Testumgebung wird der Quelltext mittels IFrame eingebunden, der Test ausgeführt und danach die Ergebnisse an den Server gesendet. Dort werden die Ergebnisse aufgearbeitet präsentiert oder automatisiert von übergeordneten Buildprozessen verarbeitet. Das verwendete Prinzip stammt aus einer Masterthesis \cite{karma-masterThesis}; tieferes Verständnis ist jedoch für die reine Nutzung von Karma nicht erforderlich.\cite{karma-howItWorks}

Für die Konfiguration wird eine JavaScript-Datei, die \texttt{karma.conf.js}, genutzt. Ein Beispiel findet sich in Listing \ref{lst:karma-conf-js}. Mit dieser beispielhaften Konfiguration wird für die Testausführung das Framework Jasmine (s. Abschnitt \ref{sec:Jasmine}) genutzt. Der Parameter \texttt{files} gibt an, welche Dateien von Karma ausgeliefert und beobachtet werden, und somit bei Änderung welcher Dateien die Tests automatisch erneut ausgeführt werden. Außerdem ist konfiguriert, dass bei Testdurchführung automatisch Firefox gestartet werden und in diesem die Tests ausgeführt werden soll.\cite{karma-configurationFile, karma-files}

\begin{figure}[H]
	\lstinputlisting[caption={Beispiel einer \texttt{karma.conf.js} (adaptiert nach \cite{karma-configurationFile, karma-files})}, label=lst:karma-conf-js]{lst/karma.conf.js}
\end{figure}

Die Funktionalität von Karma lässt sich mit Plugins erweitern. Sie werden für die Einbindung von Testframeworks, ein verändertes Ausgabeformat der Testergebnisse, Präprozessoren (z.\,B. für die Auslieferung von in JavaScript eingebettetem HTML oder die Ermittlung der Code Coverage)\cite{karma-preprocessors} oder die Einbindung von Browsern wie Firefox, Chrome oder PhantomJS (s. Abschnitt \ref{sec:PhantomJS}) benötigt. Jedes Plugin ist ein npm-Paket, daher werden Plugins über npm installiert. Karma bindet alle installierten Pakete mit dem Namen \texttt{karma-*} automatisch ein.\cite{karma-plugins} Im npm-Repository liegen über 1300 Karma-Plugins.\cite{karma-npm}

\subsubsection{Mocha}
\label{sec:Mocha}
Mocha ist ein Testframework und Test-Runner, welches sowohl in Node.js als auch in Browsern lauffähig ist. Es liegt als Paket \texttt{mocha} im npm-Repository und kann darüber installiert werden.\cite{mocha-index}

Tests bestehen in Mocha aus drei Ebenen. Die Oberste sind die Testsuites, welche weitere Testsuites oder Testfälle enthalten können. Testfälle bestehen aus funktionalem Code sowie Assertions als eigentliche Testüberprüfung. Es werden verschiedene Stile zur Testbeschreibung unterstützt: BDD, TDD, QUnit und weitere, welche sich nur in ihrem Aussehen unterscheiden und Entwicklern ermöglichen, ihren eigenen Stil zur Definition von Tests zu wählen.\cite{mocha-index}

Für Assertions können in Mocha verschiedene Frameworks genutzt werden. In \cite{mocha-index} wird beispielsweise die Nutzung von \textit{should.js} bei Verwendung des BDD-Stils, \textit{expect.js} oder \textit{chai} (s. Abschnitt \ref{sec:Chai}) empfohlen. Es ist einem Entwickler somit möglich, Mocha auf die eigenen Vorlieben anzupassen.\cite{mocha-index} Auf eine genauere Vorstellung und Codebeispiele wird an dieser Stelle aufgrund der Vielseitigkeit verzichtet.

Mocha ermöglicht es, die Testausgabe beliebig zu konfigurieren, so dass beispielsweise eine Ausgabe in der Konsole, als HTML-Datei, als JSON oder im XML-Format möglich ist. Bei besonderen Anforderungen können eigene Reporter erstellt werden, z.\,B. zur Einbindung in Continuous-Integration-Tools.

\subsubsection{AVA}
\label{sec:Ava}
AVA ist ein Test-Runner für in JavaScript geschriebene Tests. Die Besonderheit ist, dass die Tests simultan ausgeführt werden und somit deutliche Performanceverbesserungen möglich sind. Es steht unter \texttt{ava} im npm-Repository zur Verfügung.\cite{ava}

AVA läuft ausschließlich in Node.js; es gibt also keine Möglichkeit den Test-Runner im Browser aufzurufen. Ein Test wird über Aufruf einer Funktion, welche aus dem Node-Modul \textit{ava} importiert wird, definiert. Dieser Funktion wird ein Funktionskörper übergeben, welcher den Test spezifiert und Assertions durchführt. Ein Beispiel findet sich in Listing \ref{lst:ava}. Die Assertion-Funktionen werden über das übergebene Testobjekt \texttt{t} bereitgestellt. Es ist nicht vorgesehen andere Assertionframeworks zu nutzen. Mit den Funktionen \texttt{before}, \texttt{after}, \texttt{beforeEach} und \texttt{afterEach} können Funktionen definiert werden, welche vor oder nach jedem Test oder einmalig allen Tests ausgeführt werden.\cite{ava}

\begin{figure}[H]
	\lstinputlisting[caption={Beispiel eines Tests mit AVA (adaptiert nach \cite{ava})}, label=lst:ava]{lst/ava.js}
\end{figure}

\subsubsection{QUnit}
\label{sec:QUnit}
QUnit ist ein Framework für automatisierte Komponententests mit JavaScript und wird von jQuery und einer Vielzahl weiterer Projekten genutzt. Es liegt unter \texttt{qunitjs} als Paket im npm-Repository. Es kann sowohl in Browsern als auch in Node.js ausgeführt werden.\cite{qunit-index}

In QUnit geschriebene Tests ähneln denen vieler Testframeworks populärer anderer Sprachen, wie beispielsweise JUnit in Java. Ein Testfall wird mittels Aufruf von \texttt{QUnit.test(string, function)} definiert. In der übergebenen Funktion kann Testcode aufgerufen werden und das Ergebnis mit Assertions validiert werden. Wenn mindestens eine Assertion fehlschlägt, gilt der Test als fehlgeschlagen; sonst als bestanden. QUnit liefert Assertions mit: beispielsweise \texttt{assert.ok}, welche einen truthy Wert erwartet, oder \texttt{assert.equal}, welches zwei als gleich angesehene Werte erwartet.\cite{qunit-cookbook}

Tests können in durch Aufruf von \texttt{QUnit.module(string)} erzeugten Modulen gruppiert werden (s. Listing \ref{lst:qunit-test}). In Modulen kann Code ausgelagert werden, indem die vor und nach jedem Test aufgerufenen Funktionen \texttt{beforeEach} und \texttt{afterEach} definiert werden.\cite{qunit-cookbook}

\begin{figure}[H]
	\lstinputlisting[caption={Beispiel mehrerer Tests für QUnit (adaptiert nach \cite{qunit-cookbook})}, label=lst:qunit-test]{lst/qunit-1.js}
\end{figure}

\subsubsection{Intern}
\label{sec:Intern}
Intern ist ein Framework für automatisierte Tests in JavaScript. Es bietet sowohl Möglichkeiten für Komponenten- als auch für End-To-End-Tests. Intern ist flexibel und bietet dem Entwickler Möglichkeiten, seinen eigenen Stil zu verfolgen. Es steht als Paket \texttt{intern} über npm zur Verfügung.\cite{intern-userguide}

Intern stellt verschiedene Interfaces zur Definierung von Tests bereit; es können auch eigene definiert werden. Das \textit{Object}-Interface ist eine einzelne Funktion, welcher ein Objekt übergeben wird, welches alle definierten Tests enthält. In diesem Objekt werden außerdem Funktionen die vor oder nach jedem Test oder der Testsuite ausgeführt werden sollen definiert. Ein Beispiel findet sich in Listing \ref{lst:intern-object}.\cite{intern-userguide}

\begin{figure}[H]
	\lstinputlisting[caption={Beispiel einer Testsuite in Intern (aus \cite{intern-userguide})}, label=lst:intern-object]{lst/intern-object.js}
\end{figure}

Die Interfaces \textit{BDD} und \textit{TDD} ähneln einander und unterscheiden sich nur durch die Benennung einzelner Funktionen. Sie verfolgen gegenüber dem Object-Interface einen prozeduraleren Ansatz, basieren also auf verschachtelt aufgerufenen Funktionen statt auf Objekten. Auch hier können Funktionen definiert werden, welche vor oder nach jedem Test oder der Suite aufgerufen werden. Ein Beispiel ist in Listing \ref{lst:intern-bdd} zu finden.\cite{intern-userguide}

\begin{figure}[H]
	\lstinputlisting[caption={Beispiel einer Testsuite in Intern (aus \cite{intern-userguide})}, label=lst:intern-bdd]{lst/intern-bdd.js}
\end{figure}

Es steht ein Interface zur Verfügung, welches QUnit nachempfindet und somit die Verwendung von in QUnit geschriebenen Tests (s. Listing \ref{lst:qunit-test} in Abschnitt \ref{sec:QUnit}) in Intern ermöglicht. Für alle Tests gilt, dass ein Test fehlschlägt, wenn in ihm ein Error auftritt, also eine Assertion fehlschlägt. Ansonsten gilt er als bestanden. Die Chai-Bibliothek ist in Intern enthalten, es ist jedoch auch die Nutzung von beliebigen anderen Assertion-Frameworks möglich. Durch Aufruf der Funktion \texttt{skip} können Tests übersprungen werden.\cite{intern-userguide}

End-To-End-Tests werden genau wie Unit-Tests definiert, werden jedoch in der Konfiguration in \texttt{functionalSuites} und nicht in \texttt{suites} aufgeführt. Hierdurch werden sie im Kontext des Test-Runners und nicht in der zu testenden Umgebung ausgeführt. Für die Interaktion mit dem Browser verwendet Intern mit \textit{leadfoot} einen Wrapper für Selenium. Über das in \texttt{this.remote} zur Verfügung gestellte Leadfoot-Command-Objekt können Befehle im Browser ausgeführt werden.\cite{intern-userguide}

\begin{figure}[H]
	\lstinputlisting[caption={Beispiel eines End-To-End-Tests in Intern (aus \cite{intern-userguide})}, label=lst:intern-e2e]{lst/intern-e2e.js}
\end{figure}

Im Beispiel in Listing \ref{lst:intern-e2e} wird ein Testfall \glqq greeting form\grqq{} definiert, in welchem die Index-Seite geladen wird und auf dieser in einem Eingabefeld der Wert \glqq Elaine\grqq{} eingegeben und der Submit-Button geklickt wird. Abschließend wird überprüft ob das Element mit der ID \glqq greeting\grqq{} den korrekten Inhalt enthält.

Intern ermittelt standardmäßig beim Ausführen von Tests die Code Coverage, also die Abdeckung von Codezeilen, Funktionen, Zweigen und Anweisungen. Die Ausgabe sowohl von Code Coverage als auch der Testergebnisse ist konfigurierbar.\cite{intern-userguide}


\subsubsection{Jasmine}
\label{sec:Jasmine}
Jasmine ist ein Behavior Driven Development Framework zum Test von JavaScript\cite{jasmine-introduction}. Es liegt unter \texttt{jasmine} im npm-Repository\cite{jasmine-getting-started}. Jasmine bietet eine saubere und einfache Syntax zur Beschreibung von Testfällen. Die Tests bestehen aus drei Ebenen: Testsuites, Spezifizierungen (\glqq Specs\grqq) und Erwartungen, also den eigentlichen Testassertions\cite{jasmine-introduction}. Ein beispielhafter Test findet sich in Listing \ref{lst:jasmine-1}.

Eine Testsuite beginnt auf oberster Ebene mit dem Aufruf der globalen JavaScript-Funktion \texttt{describe(string, function)}. Der String ist hierbei der Name der Testsuite, üblicherweise wird hier das Testsubjekt benannt. Die Funktion implementiert die Testsuite und besteht aus Specs.\cite{jasmine-introduction}

Ein Spec wird durch Aufruf der globalen Funktion \texttt{it(string, function)} angelegt. Der String enthält eine Beschreibung des Testfalls; nach dem BDD-Modell also eine Beschreibung des erwarteten Verhaltens. Die Funktion dient zum Überprüfen dieses Verhaltens und enthält Assertions, welche entweder \texttt{true} oder \texttt{false} ergeben. Liefern alle Assertions \texttt{true} so gilt die Spec als bestanden, ansonsten als durchgefallen.\cite{jasmine-introduction}

Eine Assertion besteht in Jasmine aus der Funktion \texttt{expect(object)}, welcher der tatsächliche Wert übergeben wird. Diese wird mit einer Matcher-Funktion verkettet, welche den erwarteten Wert übergeben bekommt und die beiden Werte vergleicht und auswertet. Es wird eine Vielzahl an vorgefertigten Matchern mitgeliefert: \texttt{toEqual}, \texttt{toContain}, \texttt{toBeTruthy}, und Weitere\cite{jasmine-introduction, jasmine-cheatsheet}.

\begin{figure}[H]
	\lstinputlisting[caption={Beispiel einer Jasmine-Testsuite (adaptiert nach \cite{jasmine-introduction})}, label=lst:jasmine-1]{lst/jasmine-1.js}
\end{figure}

Specs können als \textit{pending} deklariert werden. Sie werden dann nicht ausgeführt, aber im Ergebnis angezeigt. Hierfür kann beim Aufruf der \texttt{it}-Funktion die Übergabe einer Funktion weggelassen werden, stattdessen die \texttt{xit}-Funktion aufgerufen werden oder im Funktionskörper die \texttt{pending}-Funktion genutzt werden.\cite{jasmine-introduction}

\subsubsection{Chai}
\label{sec:Chai}
Chai ist eine Assertion-Bibliothek, welche mit jedem Testframework kombiniert werden kann. Chai bietet verschiedene Assertion-Stile, so dass der Entwickler seinen eigenen Stil wählen kann.\cite{chai-index} Chai ist unter der ID \texttt{chai} über npm verfügbar und kann so installiert und in Projekte eingebunden werden\cite{chai-installation}.

Der Assert-Stil ähnelt klassischeren Testframeworks wie QUnit oder dem Assert-Modul in Node.JS. Über das \texttt{assert}-Objekt werden Funktionen zur Verfügung gestellt, zum Beispiel \texttt{isOk} zur Überprüfung, ob der Parameter truthy ist oder \texttt{equal} zur Überprüfung, ob die Parameter gleich sind. Jeder Funktion kann eine optionale Nachricht übergeben werden, welche im Falle eines Fehlschlags in der Fehlermeldung angezeigt wird.\cite{chai-assert}

Der BDD-Stil wird in zwei Varianten zur Verfügung gestellt: \texttt{expect} und \texttt{should}. Er ermöglicht es Assertions in einer natursprachlichen Form zu schreiben, welche somit gut lesbar sind. Die should-Variante hat einige Nachteile, weshalb an dieser Stelle nur die expect-Variante betrachtet wird. \texttt{Expect} ist eine Funktion, welcher der zu überprüfende Wert übergeben wird. Diese Methode wird mit weiteren Objekten und Funktionen verkettet, um die Assertion zu bilden.\cite{chai-styles} Ein Beispiel hierzu findet sich in Listing \ref{lst:chai-expect}.

\begin{figure}[H]
	\lstinputlisting[caption={Beispiel von Assertions mit dem expect-Stil von Chai (aus \cite{chai-styles})}, label=lst:chai-expect]{lst/chai-expect.js}
\end{figure}

\subsubsection{Protractor}
\label{sec:Protractor}
Protractor ist ein speziell für Angular-Anwendungen entwickeltes Framework für End-to-End-Tests. Die Tests werden in Browsern direkt gegen die Anwendungsoberfläche durchgeführt und simulieren somit das Verhalten eines echten Benutzers. Es liegt im npm-Repository mit der ID \texttt{protractor} und ist dadurch einfach zu installieren.\cite{protractor-index}

Für die Steuerung des Browsers greift Protractor auf Selenium zurück\cite{protractor-index}, welches den W3C WebDriver-Standard implementiert und als Proxyserver zwischen Protractor und dem Browser agiert\cite{selenium}. Selenium unterstützt alle großen Webbrowser: aktuell die aktuellsten Versionen von Firefox, Internet Explorer ab Version 7, Safari ab Version 5.1, Opera und Chrome\cite{selenium-browsers}. Vom Einsatz von PhantomJS (s. Abschnitt \ref{sec:PhantomJS}) zusammen mit Protractor wird ausdrücklich abgeraten, da es hier Berichten zufolge häufig zu Abstürzen und abweichendem Verhalten kommt\cite{protractor-browser}. Laut eigener Aussage wird Selenium automatisch zusammen mit Protractor installiert und ist nach Aufruf von \texttt{webdriver-manager update} und \texttt{webdriver-manager start} ohne weitere Konfiguration lauffähig\cite{protractor-index}.

Protractor nutzt als Framework für die Testbeschreibung standardmäßig Jasmine (s. Abschnitt \ref{sec:Jasmine}), unterstützt out-of-the-box aber auch Mocha (s. Abschnitt \ref{sec:Mocha}). Die nachfolgenden Beispiele nutzen daher auch Jasmine. Das eingesetzte Testframework, die Adresse unter welcher der Selenium-Server angesprochen wird, Testdateien, Timeouts, für den Test zu verwendende Browser und weitere Feineinstellungen werden in einer Konfigurationsdatei (s. Listing \ref{lst:protractor-conf}) konfiguriert.

\begin{figure}[H]
	\lstinputlisting[caption={Beispiel einer Protractor-Konfiguration (adaptiert nach \cite{protractor-tutorial})}, label=lst:protractor-conf]{lst/protractor.conf.js}
\end{figure}

Üblicherweise hat jede zu testende Seite eine eigene Testsuite und jeder Testfall ist eine eigene Spec (s. Listing \ref{lst:protractor-spec}). Vor der eigentlichen Testdurchführung muss die jeweilige Seite aufgerufen werden: hierzu dient die durch Protractor bereitgestellte Funktion \texttt{browser.get(url)}. Es bietet sich an, diese in \texttt{beforeEach()} auszuführen, einer Funktion die durch Jasmine vor jedem Spec aufgerufen wird. Auf Elemente kann mit der Funktion \texttt{element} zugegriffen werden, welcher ein Locator übergeben wird. Locator sind ein durch Protractor definiertes Konstrukt und beschreiben, wie das Element gefunden werden kann. Um mit den gefundenen Elementen zu interagieren werden verschiedene Funktionen bereitgestellt: beispielsweise \texttt{sendKeys} zur Zeicheneingabe, \texttt{click} zum Simulieren eines Mausklicks oder \texttt{getText} um den Elementinhalt zu ermitteln.

\begin{figure}[H]
	\lstinputlisting[caption={Beispiel einer Spec für Protractor (aus \cite{protractor-tutorial})}, label=lst:protractor-spec]{lst/protractor.spec.js}
\end{figure}





\subsubsection{PhantomJS}
\label{sec:PhantomJS}
PhantomJS ist ein skriptbarer WebKit-Browser ohne Benutzeroberfläche und ist über eine JavaScript-API ansteuerbar. Er bietet native Unterstützung für Webstandards wie DOM, CSS-Selektoren, JSON, HTML-Canvas und SVG.\cite{phantomjs-index} PhantomJS steht nicht als npm-Paket zur Verfügung, sondern lässt sich lediglich als Binary installierten\cite{phantomjs-faq}. PhantomJS ist auch eine Laufzeitumgebung für JavaScript, so dass für ihn bestimmte Skripte nicht in Node.js ausgeführt werden können, sondern nur in PhantomJS\cite{phantomjs-quickstart}.

Für das Laden von Webseiten sind Page-Objekte zuständig, über welche eine URL geladen werden kann, Screenshots gespeichert werden können oder auf DOM-Eigenschaften zugegriffen werden kann. Ein Beispiel findet sich in Listing \ref{lst:phantomjs}. JavaScript-Code kann im Kontext einer geladenen Seite mit der \texttt{evaluate}-Funktion ausgeführt werden; die Ausführung erfolgt in einer Sandbox und kann somit nicht auf Objekte, Variablen oder Funktionen außerhalb des Kontexts zugreifen.\cite{phantomjs-quickstart}

\begin{figure}[H]
	\lstinputlisting[caption={Beispiel eines Seitenaufrufs mit PhantomJS (aus \cite{phantomjs-quickstart})}, label=lst:phantomjs]{lst/phantom.js}
\end{figure}

Ein großer Nutzungsbereich von PhantomJS liegt im Testen von Webanwendungen; geeignet ist es beispielsweise für den Einsatz in Kommandozeilenumgebungen oder Continuous-Integration-Systemen. PhantomJS an sich ist kein Testframework, sondern dient der Ausführung eines beliebigen Testframeworks. Es existieren Frameworks welche speziell auf PhantomJS aufbauen und komfortable Funktionalitäten für Testzwecke zur Verfügung stellen: z.\,B. CasperJS (s. Abschnitt \ref{sec:CasperJS}) oder WebSpecter, welches sich allerdings noch in einer frühen Entwicklungsphase befindet.\cite{phantomjs-testing}

\subsubsection{CasperJS}
\label{sec:CasperJS}
CasperJS ist ein Framework für Navigation und Test in PhantomJS. Es steht unter \texttt{casperjs} im npm-Repository zur Verfügung.\cite{casperjs-index} Trotz dessen ist es nicht unter Node.js lauffähig\cite{casperjs-faq}, sondern benötigt Python\cite{casperjs-installation}.

Es ermöglicht die Erstellung von komplexen Navigationsszenarien unter Benutzung von High-Level-Funktionen. Hierzu werden u.\,a. die Funktionen \texttt{casper.start}, \texttt{casper.then}, \texttt{casper.thenOpen} und \texttt{casper.back} sequentiell aufgerufen. Das Szenario kann dann mittels \texttt{casper.run} aufgerufen werden und wird nacheinander abgearbeitet. Es stehen diverse Funktionen wie \texttt{casper.click}, \texttt{casper.fill} oder \texttt{casper.evaluate} zur DOM-Manipulation zur Verfügung. Für alle Browser-Operationen und DOM-Manipulationen wird als Browser PhantomJS genutzt, welcher von CasperJS gestartet wird. Die Ansteuerung von PhantomJS wird durch CasperJS vereinfacht, wodurch sich einfacher wartbarerer Code ergibt\cite{casperjs-better-phantomjs}. Ein beispielhaftes Navigationsszenario findet sich in Listing \ref{lst:casper-scenario}.\cite{casperjs-index}

\begin{figure}[H]
	\lstinputlisting[caption={Beispiel eines Navigationsszenarios mit CasperJS (adaptiert nach \cite{casperjs-index})}, label=lst:casper-scenario]{lst/casper-scenario.js}
\end{figure}

CasperJS enthält auch ein einfaches Testframework. Ein Test wird durch Aufruf der Funktion \texttt{casper.test.begin} definiert und mit \texttt{test.done} beendet. Er gilt als erfolgreich, wenn er ohne fehlgeschlagene Assertions beendet wurde. Navigationsszenarien und Tests können verschachtelt werden, so dass mit CasperJS auch End-To-End-Tests durchgeführt werden können.\cite{casperjs-index, casperjs-test} Ein einfacher Test ist in Listing \ref{lst:casper-test} zu finden.

\begin{figure}[H]
	\lstinputlisting[caption={Beispiel eines Tests mit CasperJS (aus \cite{casperjs-test})}, label=lst:casper-test]{lst/casper-test.js}
\end{figure}


\subsubsection{Sinon}
\label{sec:Sinon}
Sinon ist ein Framework für die Realisierung von Fakes, also Mocks, Stubs und Spies, in JavaScript und arbeitet mit jedem Unit-Test-Framework zusammen. Es ist als \texttt{sinon} über npm verfügbar.\cite{sinonjs-index}

\paragraph*{Spies}
Spies sind Funktionen die relevante Aufrufdaten aufzeichnen: Argumente, Rückgabewert, geworfene Exceptions und den Aufrufer. Ein Spy kann sowohl als anonyme Funktion - durch Aufruf von \texttt{sinon.spy()} - als auch als Wrapper für existierende Methoden - dann durch Aufruf von \texttt{sinon.spy(object, 'method')} für das Überschreiben von \texttt{object.method()}.\cite{sinonjs-spies} Ein Beispiel für einen Spy auf einer anonymen Funktion findet sich in Listing \ref{lst:sinon-spy}.

\begin{figure}[H]
	\lstinputlisting[caption={Beispiel eines anonymen Spy in Sinon (aus \cite{sinonjs-index})}, label=lst:sinon-spy]{lst/sinon-spy.js}
\end{figure}

\paragraph*{Stub}
Stubs sind Spies mit einem vorprogrammierten Verhalten. Hierzu verfügen sie über die komplette Spy-API und zusätzliche Methoden, mit welchen ihr Verhalten angepasst werden kann. Anders als bei Spies wird bei einem Stub, welcher eine existierende Funktion überschreibt, diese nicht ausgeführt. Sie können benutzt werden um ein bestimmtes Verhalten von Funktionen zu provozieren, z.\,B. durch das Werfen von Fehlern, oder um zu verhindern, dass Funktionen ausgeführt werden, z.\,B. \texttt{XMLHttpRequest} damit kein HTTP-Request abgesetzt wird. Ein Aufruf von \texttt{sinon.stub().throws()} erzeugt beispielsweise einen anonymen Stub, welcher bei Aufruf eine Exception wirft.\cite{sinonjs-stubs}

\paragraph*{Mocks}
Mocks sind Stubs, welche zusätzlich vorprogrammierte Erwartungen haben. Ein Mock ist somit ein Stub, welcher Assertions enthält. Es wird empfohlen, dass maximal ein Mock pro Unittest existiert. Ein Beispiel für einen Mock, in welchem erwartet wird, dass die gemockte Methode einmal aufgerufen wird und diese eine Exception werfen soll, findet sich in Listing \ref{lst:sinon-mock}. Das Eintreffen der definierten Erwartungen wird letztlich durch Aufruf von \texttt{verify()} überprüft - ein Nicht-Zutreffen führt wie bei Assertions zum Fehlschlag des Tests.\cite{sinonjs-mocks}

\begin{figure}[H]
	\lstinputlisting[caption={Beispiel eines Mocks in Sinon (aus \cite{sinonjs-mocks})}, label=lst:sinon-mock]{lst/sinon-mock.js}
\end{figure}

Sinon ermöglicht es, mit der Funktion \texttt{sinon.restore} alle Fakes, die auf einem übergebenen Objekt definiert wurden, zurückzusetzen. Es bietet außerdem die Möglichkeit \textit{Sandboxes} anzulegen, in welchen alle angelegten Fakes abgelegt werden. Dies erleichtert das Aufräumen und Entfernen aller Fakes, da dies nun nicht mehr einzeln geschehen muss, sondern ein Aufruf von \texttt{sandbox.restore} genügt.\cite{sinonjs-sandboxes}

\subsubsection{ngMock}
\label{sec:ngMock}
Bei ngMock handelt es sich um ein AngularJS-Modul, mit welchem andere Komponenten in Unit-Tests injiziert und gemockt werden können. Außerdem erweitert es diverse AngularJS-Kernservices, so dass diese in Testcode kontrolliert und inspiziert werden können. Es steht im npm-Repository unter \texttt{angular-mocks} zur Verfügung und muss in der Konfiguration des verwendeten Test-Runners so eingebaut werden, dass es nach \texttt{angular.js} geladen wird.\cite{angular-ngMock}

Die Funktionsweise basiert auf den Methoden \texttt{angular.mock.module}, welche die übergebenen Module lädt\cite{angular-ngMockModule}, und \texttt{angular.mock.inject}, welche eine übergebene Funktion in eine Injizierbare wrapt, eine neue Injector-Instanz erstellt und die angegebenen Abhängigkeiten injiziert\cite{angular-ngMockInject}. Methoden von injizierten Abhängigkeiten können, beispielsweise mit Sinon (s. Abschnitt \ref{sec:Sinon}), mit Spies oder Mocks ersetzt werden.

\begin{figure}[H]
	\lstinputlisting[caption={Beispiel eines Tests mit injizierten Abhängigkeiten mit ngMock (adaptiert nach \cite{watmore})}, label=lst:ngMock]{lst/ngMock.js}
\end{figure}

Im Beispiel (s. Listing \ref{lst:ngMock}) wird das Module \texttt{app} geladen und die Services \texttt{SimpleService} und \texttt{\$log} in die Testfunktion injiziert. Dadurch kann ein Spy auf \texttt{\$log.info} gesetzt werden und hierdurch das korrekte Logging von \texttt{SimpleService.DoSomething} validiert werden.





\newpage
\subsection{Auswahlentscheidung}
\label{sec:auswahlentscheidung}
Für die Entscheidungsfindung wird eine Entscheidungsmatrix (s. Tabelle \\ref{tbl:entscheidungsmatrix}) erstellt, in welcher die gefundene Software mit Blick auf die Anforderungen bewertet wird. Ein rotes Kreuz bedeutet, dass die Software der Anforderung widerspricht und die Erfüllung der Anforderung unmöglich macht. Ein grüner Haken steht für ein Erfüllen der Anforderung. Eine leere Zelle bedeutet, dass die Software die Anforderung weder explizit erfüllt noch ihr widerspricht; es besteht keine Relevanz. Die ausgewählte Softwaremenge erfüllt eine Anforderung somit nur dann, wenn lediglich grüne Haken oder leere Zellen vorhanden sind, also keine Einzelsoftware der Anforderung widerspricht.

Wie in Abschnitt \ref{sec:anforderungsanalyse} bereits beschrieben werden zur Erfüllung von Anforderung A1 ein Testrunner, ein Testframework sowie eine Assertion-Bibliothek benötigt; zur Erfüllung von Anforderung A2 zusätzlich ein Tool zur Browsersteuerung. Hierzu werden die gefundenen Programme kategorisiert. Die endgültig ausgewählte Softwaremenge muss mindestens ein Exemplar jeder Kategorie enthalten, damit die Anforderungen A1 und A2 erfüllt werden.

%\begin{table}[H] \centering
	%\begin{tabular}{@{} cl*{11}c @{}}
	%	& & \rot{Karma} & \rot{Mocha} & \rot{AVA} & \rot{QUnit} & \rot{Intern} & \rot{Jasmine} & \rot{Chai} & \rot{Protractor} & \rot{PhantomJS} & \rot{CasperJS} & \rot{Sinon} & \rot{ngMock} \\\hline
	%	& \textit{1a} - Komponententest & \OK & \OK & \OK & \OK & \OK & \OK & & & & & &  \\
	%	
	%\end{tabular}
	\begin{tabularx}{\textwidth}{@{}r@{\hskip 6pt}X|ccccccccccccc}
		&& \multicolumn{12}{c}{Software} \\ &&  \rot{\footnotesize{Karma}} & \rot{\footnotesize{Mocha}} & \rot{\footnotesize{AVA}} & \rot{\footnotesize{QUnit}} & \rot{\footnotesize{Intern}} & \rot{\footnotesize{Jasmine}} & \rot{\footnotesize{Chai}} & \rot{\footnotesize{Protractor}} & \rot{\footnotesize{PhantomJS}} & \rot{\footnotesize{CasperJS}} & \rot{\footnotesize{Sinon}} & \rot{\footnotesize{ngMock}} & \rot{\footnotesize{Istanbul}} \\ \hline
		%									Ka		Mo		Av		QU		In		Ja		Ch		Pr		Ph		Ca		Si		ng		Is
	    \multirow{5}{*}{\centering\rot{\footnotesize{Kategorie}}}
		&	\footnotesize{Testrunner}	& \ok	& \ok	& \ok	& \ok	& \ok	& \ok	& 		& 		&		& \ok	&		&  		&\\
	    &	\footnotesize{Testframework}
							    		&		& \ok	& \ok	& \ok	& \ok	& \ok	&		&		&		& \ok	&		& 		&\\
	    &	\footnotesize{End-To-End}	
									    & 		&		&		&		& \ok	&		&		& \ok	& \ok	& \ok	&		&  		&\\
	    &	\footnotesize{Assertion}	& 		&		& \ok	& \ok	& \ok	& \ok	& \ok	& 		& 		& \ok	& 		&  		&\\
	    &	\footnotesize{Sonstige}		&		&		&		&		&		&		&		&		&		&		& \ok	& \ok	& \ok \\\hline\hline
	    
	    \footnotesize{A3}
	     &	\footnotesize{AngularJS-Support}	
									    & \ok	& \ok	& \nok\footnote{\url{https://github.com/angular/angular.js/issues/13971}}
																& \ok	& \ok	& \ok	& \ok	& \ok	& \ok	& \ok
																													    & \ok	& \ok	& \ok \\
		\footnotesize{A4}
		 & \footnotesize{Testausführung}& \ok	& 		& 	  	& 		& 		& \ok	& 		&		&		& \nok	&		&		& \\
		
		\footnotesize{A5}
		 & \footnotesize{Open Source \& kommerziell}
										& \ok	& \ok	& \ok	& \ok	& \ok	& \ok	& \ok	& \ok	& \ok	& \ok	& \ok	& \ok	& \ok \\
		\footnotesize{A6}
		 & \footnotesize{Wartung\footnote{Zur Auswertung dieser Anforderung wird das jeweilige Github-Repository herangezogen.}}
										& \ok	& \ok	& \ok	& \ok	& \ok	& \ok	& \ok	& \ok	& \nok\footnote{1.841 offene Issues; 2 Issues geschlossen in der letzten Woche \cite{phantomjs-issues,phantomjs-pulse}}
																												& \ok	& \ok	& \ok	& \ok \\ 
		%11 & \textbf{ECMA-262 5.1}		& 		& \ok	&		&		&		&		& \ok	&		&		&		&		& \\

		\footnotesize{A7}
			 & \footnotesize{BDD-Stil}	&		& \ok	& \nok	& \nok	& \ok	& \ok	& \ok	&		&		& \nok	&		& 		&\\
		\footnotesize{A8} 
		& \footnotesize{Einarbeitung \& Verwendung}
										& \ok	& \ok	& \ok	& \ok	& \nok\footnote{Warten auf asynchrone AngularJS-Komponenten muss bei End-To-End-Tests manuell implementiert werden.}
																				& \ok	& \ok	& \ok	& \ok	& \nok\footnote{Warten auf asynchrone AngularJS-Komponenten muss manuell implementiert werden.}
																														& \ok	& \ok	& \ok \\
		\footnotesize{A9}
		 & \footnotesize{Flexibilität}& 		& \ok	& \nok	& \nok	& \ok	& \nok	& 		&		&		& \nok	&		& 		& \\
		\footnotesize{A10}
		 & \footnotesize{Ausgabe-Konfiguration}
										& \ok	& \ok	& \ok	& \ok	& \ok	& \ok	&		&		& 		& \nok	&		& 		& \ok \\
		\footnotesize{A11}
		 & \footnotesize{Screenshots}	&		&		&		&		& \nok	&		&		& \ok	& \ok	& \ok	&		& 		&\\
		\footnotesize{A12}
		 & \footnotesize{Spies, Stubs, Mocks}
										&		&		&		&		&		&		&		&		&		& 		& \ok	& \ok	& \\
		\footnotesize{CB1}
		&	\footnotesize{Windows 7}			& \ok	& \ok	& \ok	& \ok	& \ok	& \ok	& \ok	& \ok	& \ok	& \ok	& \ok	& \ok 	& \ok \\
		\footnotesize{CB2}
		& \footnotesize{benötigte Software}	& \ok	& \ok	& \ok	& \ok	& \ok	& \ok	& \ok 	& \ok\footnote{Zur Ausführung wird zwar Selenium benötigt, dies wird aber automatisiert mitinstalliert. Die Software erfüllt die Anforderung daher.}
		& \ok	& \nok	& \ok	& \ok 	& \ok \\
		\footnotesize{CB3}
		& \footnotesize{Node.js 9.6.2}& \ok	& \ok	& \ok 	& \ok	& \ok	& \ok	& \ok	& \ok	&  \ok	& \ok	& \ok	& \ok	& \ok \\
		
		\footnotesize{CB4}
		& \footnotesize{npm}			& \ok	& \ok	& \ok	& \ok	& \ok	& \ok	& \ok	& \ok	& \nok	& \ok	& \ok	& \ok	& \ok \\		
		\footnotesize{CB5}
		& \footnotesize{Forcierte Beschreibung}
		&		& \ok	& \nok	& \ok	& \ok	& \ok	& 		& 		&		& \ok	& 		&		& \\		
				\footnotesize{CB6}
				& \footnotesize{Code-Coverage}		
				& 		& 		& 		& 		& \ok	&		&		&		&		&		&		& 		& \ok\\ 
				\footnotesize{CB7}
				& \footnotesize{keine proprietären Dateien}
				& \ok	& \ok	& \ok	& \ok	& \ok	& \ok	& \ok	& \ok	& \ok	& \ok	& \ok	& \ok	& \ok	 
					
	\end{tabularx}
	\captionof{table}{Entscheidungsmatrix zur Softwareauswahl\label{tbl:entscheidungsmatrix}}
	
	Die Auswertung der Entscheidungsmatrix zeigt, dass AVA, QUnit, Intern, Jasmine, PhantomJS sowie CasperJS nicht eingesetzt werden können, da sie Anforderungen widersprechen. Die Verbleibenden bilden die Basis für die zu treffende Auswahl aufgrund der Anforderungen. Es zeigt sich, dass nur durch den Einsatz sämtlicher verbleibender Software sämtliche Anforderungen abgedeckt werden können. Es wird somit folgende Auswahl getroffen:
	
	Als Test-Runner wird Karma eingesetzt, welches die benötigten JavaScript-Dateien importiert, bereitstellt und aus der Konsole heraus die in Mocha implementierten Tests ausführt. Die sich aus der Matrix ergebende Kombination von Mocha mit der Assertionbibliothek Chai wird von \textcite{mocha-index} bereits in der Softwaredokumentation aufgeführt, wodurch hier von einem problemlosen Zusammenarbeiten auszugehen ist. Für die Ermittlung der Code-Coverage wird Istanbul eingesetzt; die Kombination des Einsatzes mit Mocha wird in  \cite{istanbul} explizit empfohlen, so dass auch hier von keinerlei Problemen auszugehen ist.
	
	Für die Realisierung von End-To-End-Tests wird auf das vom AngularJS-Team entwickelte Protractor zurückgegriffen und somit auch der Empfehlung aus der offiziellen AngularJS-Dokumentation  \cite{angular-e2e} gefolgt. Die abweichende Kombination von Protractor mit Mocha und Chai, statt Jasmine, wird von vielen Entwicklern genutzt und unter Anderem in  \cite{protractormocha-1} und  \cite{protractormocha-2} beschrieben, so dass auch hier von keinen Problemen auszugehen ist.
	
	Für die Realisierung von Spies, Stubs und Mocks in Komponententests wird auf Sinon und ngMock gesetzt.
%\end{table}