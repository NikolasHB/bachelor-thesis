AngularJS ist ein von Google Inc. ins Leben gerufenes JavaScript-Framework zur Entwicklung von clientseitigen Webanwendungen\cite{angular-faq}. Der Quellcode von AngularJS steht auf Github zur Verfügung und wird dort auch von einer großen Entwicklergemeinschaft weiterentwickelt\cite[][9]{ng-book}. Da er unter der MIT-Lizenz veröffentlicht ist eignet sich AngularJS auch für den kommerziellen Einsatz\cites[9]{ng-book}[]{mit-license}.

Ende 2016 wurde eine neue Version von Angular veröffentlicht: Angular2\cite{ng2-out}. Durch die Bezeichnung kann AngularJS (Version 1) klar von Angular2 (Version 2) abgegrenzt werden. Im Rahmen dieser Bachelorarbeit wird AngularJS betrachtet.

\subsubsection{Funktionsweise}
Bei der Entwicklung von Webanwendungen mit AngularJS kommt das Model-View-ViewModel-Entwurfsmuster (MVVM), eine Erweiterung von Model-View-Controller (MVC), zum Einsatz. 

\subsubsection{Architektur}


\subsubsection{Dependency Injection}