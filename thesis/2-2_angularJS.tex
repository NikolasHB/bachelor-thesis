AngularJS ist ein von Google ins Leben gerufenes JavaScript-Framework zur Entwicklung von clientseitigen Webanwendungen\cite{angular-faq}. Der Quellcode von AngularJS steht auf Github zur Verfügung und wird dort von einer großen Entwicklergemeinschaft weiterentwickelt\cite[][9]{ng-book}. Da es unter der MIT-Lizenz veröffentlicht ist eignet sich AngularJS auch für den kommerziellen Einsatz\cites[9]{ng-book}[]{mit-license}.

Ende 2016 wurde eine neue Version von Angular veröffentlicht: Angular2\cite{ng2-out}. Durch die Bezeichnung kann AngularJS (Version 1) klar von Angular2 (Version 2) abgegrenzt werden. Im Rahmen dieser Bachelorarbeit wird AngularJS betrachtet.

\subsubsection{Architektur}
Bei der Entwicklung von Webanwendungen mit AngularJS kommt das Model-View-ViewModel-Entwurfsmuster (MVVM), eine Erweiterung von Model-View-Controller (MVC), zum Einsatz\cite[][21]{angular-boehm}. 

Die Model-Schicht, also die Datenhaltung und Geschäftslogik, liegt hierbei auf dem Server und wird durch REST- oder WebSocket-Verbindungen dargestellt. Hierzu kommen in AngularJS meist Services zum Einsatz: Vordefinierte, wie z.\,B. der http-Service für HTTP-Abfragen, oder Selbsterstellte\cite{angular-services}. Mittels dieser kann Geschäftslogik auch clientseitig umgesetzt werden.\cite[][21]{angular-boehm}

Es ist erforderlich, die über die Model-Schicht ermittelten Daten zu verwalten und gegebenenfalls zu transformieren um sie der Anzeige zur Verfügung zu stellen. Hierfür wird die ViewModel-Schicht genutzt. Außerdem wird in dieser Schicht die Funktionalität definiert, welche die View-Schicht steuert und diese zur Kommunikation mit der Model-Schicht nutzt. Dabei handelt es sich um Funktionen zur Behandlung von Events, wie Buttonclicks, Texteingaben, etc. Zur Weitergabe der Daten an die Anzeige wird Two-Way-Databinding (s. Abschnitt \ref{sec:twoWayDatabinding}) verwendet. Umgesetzt wird die ViewModel-Schicht mit Controllern und sogenannten Scopes (s. Abschnitt \ref{sec:scopes}). \cite[][21\psq]{angular-boehm}

Die View-Schicht wird in AngularJS mit Templates und Direktiven umgesetzt. Templates sind HTML-Dateien, in welchen zusätzliche Tags und Attribute, die sogenannten Direktiven, verwendet werden. \cite[][1\psqq]{angular-boehm}Direktiven ermöglichen es wiederverwendbare Komponenten zu erschaffen, indem Template und Quelltext in einem neuen Tag oder Attribut gekapselt werden\cite[][49\psq]{angular-boehm}.

\begin{figure}[H]
\lstinputlisting[caption={Beispiel eines AngularJS-Templates, adaptiert nach \cite{angular-boehm}}, label=lst:angularTemplate]{lst/angularTemplate.html}
\end{figure}

Im Beispieltemplate (s. Listing \ref{lst:angularTemplate}) wird ein Eingabefeld (HTML \texttt{input}) definiert, dessen Inhalt automatisch mit der im Scope liegenden Variable \texttt{someModelField} synchronisiert wird. Ein \texttt{h1}-Element zeigt den Inhalt dieser Variablen an. Für beide Synchronisierungen wird automatisch Two-Way-Databinding (s. Abschnitt \todo{ref}) genutzt. Weiterhin wird ein Button definiert, welcher bei Click die Controller-Funktion \texttt{setName()} aufrufen soll. Die Angabe \texttt{ng-app} im \texttt{html}-Tag gibt das AngularJS-Modul an, welches von der Anwendung verwendet werden soll. Die Angabe \texttt{ng-controller} spezifiziert den von diesem Template zu verwendenden Controller (s. Listing \ref{lst:angularTestCtrl}).

\begin{figure}[H]
	\lstinputlisting[caption={Beispielhafter AngularJS-Controller, adaptiert nach \cite{angular-controller}}, label=lst:angularTestCtrl]{lst/angularTestCtrl.js}
\end{figure}

In der JavaScript-Datei wird im verwendeten Modul eine Funktion, die als Controller mit dem Namen \texttt{TestCtrl} dient, definiert. Dieser Controller spezifiziert die Funktion \texttt{setName}, welche dadurch im Template verwendet werden kann. Das Skript muss über Dateikonkatenation (npm-Package \glqq concat\grqq \cite{npm-concat}) oder zusätzliches Einbinden in das Template an den Browser ausgeliefert werden.

\subsubsection{Two-Way-Databinding}
\label{sec:twoWayDatabinding}
Two-Way-Databinding ist die Datenbindung in beide Richtungen. Es dient der Aktualisierung der Model-Daten anhand von Benutzereingaben in der Ansicht sowie die Anpassung und Aktualisierung der View bei Änderungen des zugrundeliegenden Datenmodells. Dieses Konzept ist integraler Bestandteil von AngularJS und erspart das Schreiben von Boilerplate-Code, der nicht zur Geschäftslogik beiträgt. Ohne Two-Way-Databinding wäre es erforderlich, auf jedem zu synchronisierenden DOM-Element einen ChangeListener zu registrieren, welcher Änderungen durch den Benutzer an das Datenmodell weiterreicht. Außerdem müsste Logik implementiert werden, welche bei einer Änderung von Variablen im Datenmodell die View aktualisiert. Die Datenbindung in AngularJS erhöht somit die Effizienz, da Programmcode mit weniger Overhead geschrieben werden kann.\cite[][24]{angular-boehm}

\subsubsection{Scopes}
\label{sec:scopes}
Scopes sind in AngularJS die Basis der Datenbindung, wobei in einem Scope die Variablen und Funktionen definiert sind, welche für einen bestimmten Teil des DOM benötigt werden. Scopes sind hierarchisch angeordnen und bilden grob die DOM-Struktur nach. Den Ursprung dieser Hierarchie bildet der Root-Scope, welcher von AngularJS standardmäßig zur Verfügung gestellt wird. Hierbei können sie entweder die Eigenschaften des jeweils übergeordneten Scopes erben oder isoliert sein. Beim Auswerten von Ausdrücken in Templates (z.\,B. \texttt{\{\{scopeVariable\}\}}) wird zuerst im mit dem jeweiligen Element assoziierten Scope und danach in den jeweils Übergeordneten nach der Eigenschaft gesucht.\cites[23\psqq]{angular-boehm}[]{angular-scopes}

Zur Erkennung, ob eine Variable im Datenmodell geändert wurde und eine Aktualisierung der Anzeige erforderlich ist, wird in AngularJS Dirty Checking genutzt. Hierzu wird von jedem Scope eine Kopie im Speicher gehalten, so dass bei jedem Event die gehaltene und aktuelle Version eines Scopes miteinander verglichen werden können. Bei veränderten Werten wird eine Aktualisierung der Anzeige angestoßen.\cites[24]{angular-boehm}[]{angular-dirty}

\subsubsection{Dependency Injection}
Dependency Injection ist ein Entwurfsmuster welches beschreibt, wie eine Komponente Zugriff auf benötigte Abhängigkeiten, also andere Komponenten, bekommt und wird in AngularJS durchgängig genutzt. Bei Nutzung von Dependency Injection werden die Komponenten nicht selber erzeugt sondern von außerhalb durch einen Injector geliefert. Hierfür ist es nötig, dass Services, Direktiven, Filter und Controller mit den entsprechenden Factory-Funktionen von AngularJS erzeugt werden. Diese registrieren einerseits die Komponente und ermöglichen es, diese in andere Komponenten zu injizieren, kümmern sich aber auch um die Bereitstellung der benötigten Komponenten. \cite{angular-di}

\begin{figure}
	\lstinputlisting[caption={Beispielhafter AngularJS-Controller mit Dependency Injection, adaptiert nach \cite{angular-di}}, label=lst:angularTestCtrlDI]{lst/angularTestCtrlDI.js}
\end{figure}

Im Beispiel in Listing \ref{lst:angularTestCtrlDI} wird im Modul \texttt{someModule} ein neuer Controller \texttt{MyController} angelegt, in welchen der Scope, sowie die zwei Services \texttt{http}, welcher standardmäßig von AngularJS zur Verfügung gestellt wird, sowie \texttt{dep}, welcher benutzerdefiniert erstellt wurde, injiziert werden. Diese Abhängigkeiten werden durch Übergabe als Funktionsparameter bereitgestellt.

Dependency Injection bietet gravierende Vorteile für die Testbarkeit. Es ermöglicht, eine Komponente durch ein spezielles selbst implementiertes Mock-Objekt zu ersetzen, dessen Verhalten festgelegt werden kann. Bei Tests kann das Verhalten der Abhängigkeiten festgelegt und Komponenten isoliert getestet werden.\cite[][27]{angular-boehm}